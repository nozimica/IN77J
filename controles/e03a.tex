\documentclass[11pt,letterpaper]{article}

\newcommand{\hmwkCode}{Control 3, primera parte}
\newcommand{\hmwkDueDate}{Entrega: mi\'{e}rcoles 16 de diciembre de 2015}
\newcommand{\hmwkClassTime}{}
\usepackage{in77j.exercise}
\usepackage{mathabx}

\setlength{\parskip}{0.7\baselineskip}
\setitemize{topsep=0pt}
\begin{document}
\begin{spacing}{1.2}
\thispagestyle{empty}
\evaluationTitle

\newcommand{\separatorLine}{\begin{center}\rule{.6\textwidth}{1pt}\end{center}}

\begin{Problem}

\bld{Considere el análisis de los sueldos de jugadores de fútbol y de tenis.}

En este problema analizaremos las similitudes y diferencias entre los jugadores de fútbol y de tenis,
teniendo como objetivo final del análisis el elaborar un diseño orientado a objetos que sirva como
base para un eventual sistema de información enfocado en analizar los sueldos anuales
de estos jugadores.

De ahora en adelante, cuando digamos \bld{deportista}, nos referiremos indistintamente a un jugador
de fútbol o de tenis.

Es importante que su modelo tenga en cuenta los puntos siguientes:
\begin{itemize}
    \item Solamente considere futbolistas y tenistas.
    \item Un futbolista estará ---en todo momento--- asociado a lo más a un equipo
        de fútbol. De estar en un equipo, tendrá asociado un contrato con un año de comienzo
        y un año de término de contrato, y con un sueldo anual fijo.
        \begin{itemize}
            \item No se preocupe por contratos que terminan a mitad de año.
        \end{itemize}
    \item De un futbolista también tendremos registro de sus ganancias acumuladas hasta el último
        contrato en que estuvo involucrado (no considerando al actual).
    \item Un tenista nunca está asociado a equipo alguno pues supondremos que siempre
        jugará de manera individual.
    \item De las ganancias de los tenistas sí deberemos saber:
    \begin{itemize}
        \item Cuánto fue lo que ganaron en total hasta el año anterior.
        \item Cuánto es lo que llevan ganado durante el año en curso.
    \end{itemize}
    \item Todo deportista tendrá un nombre, año de nacimiento, año en que comenzó a jugar profesionalmente
        y el país al que está asociado.
    \item Estos deportistas son representados por un \itt{manager}, del cual nos interesará
        saber su nombre y el listado de deportistas que representa.
    \item De los futbolistas nos interesa saber cuántos partidos han jugado por la selección de su país (da
        lo mismo aquellos partidos donde estuvieron todo el tiempo en la banca de suplentes).
    \item De los tenistas nos interesa saber cuántos partidos (singles o dobles) han jugado en la Copa Davis
        representando a su país.
\end{itemize}

En base a la información anterior, su modelo debe proveer operaciones para cubrir los cálculos listados a continuación.
Tenga presente que su modelo no debe realizar estos cálculos todavía, solamente debe declarar cómo se
llamará la operación que entregue dicho cálculo y los parámetros que podrían ser relevantes.

\begin{itemize}
    \item Para cada deportista, necesitamos calcular las ganancias totales que acumulará hasta diciembre del
        año actual (sea cual sea este año).
    \item Para cada deportista, necesitamos saber si alguna vez ha representado a su país.
    \item Para cada equipo de fútbol, necesitamos saber su listado de jugadores.
    \item Para cada equipo de fútbol, necesitamos saber cuánto están pagando en total por concepto
        de contratos a sus jugadores.
    \item Por cada \itt{manager}, queremos saber el promedio de sueldo anual de sus jugadores de fútbol.
    \item Por cada \itt{manager}, queremos saber el promedio de ganancias acumuladas hasta el año anterior
        para sus jugadores de tenis.
\end{itemize}

\begin{ProbPart}%[difficulty={10}]

    Diseñe un diagrama de dominio que modele adecuadamente esta situación.
\end{ProbPart}

\begin{ProbPart}%[difficulty={10}]

    Diseñe un diagrama de clases completo, que modele adecuadamente esta situación, considerando como mínimo
    todos los atributos mencionados en la descripción anterior. Si bien usted no debe mencionar
    en su diagrama ni los \itt{setters} ni \itt{getters}, sí incluya todas aquellas operaciones que
    permitan obtener toda información específicamente solicitada en la descripción anterior.

\end{ProbPart}

\begin{ProbPart}%[difficulty={10}]

    ¿Qué agregaría en sus diagramas anteriores si ahora se le solicita considerar a entrenadores de equipos de fútbol?  
    Para este caso suponga que:
\begin{itemize}
    \item Todo entrenador de fútbol fue anteriormente un jugador profesional.
    \item Todo entrenador está asociado a lo más a un equipo de fútbol o a la selección de algún país.
    \item En ambos casos, suponga que el contrato entre un entrenador y un equipo (o selección) tiene las mismas características 
        que un contrato para los jugadores profesionales.
    \item De un entrenador también tendremos registro de las ganancias que acumuló (sólo como entrenador) hasta el
        último contrato no considerando al actual.
\end{itemize}

\end{ProbPart}

\end{Problem}

\begin{NewAnswer}

\begin{ProbPart}

\begin{center}
\begin{tikzpicture}
    \umlclass[x=0,y=0]{Deportista}{-- nombre\\-- añoNac\\-- añosJugadorProf\\-- país}{}
    \umlclass[x=-6,y=-6]{Manager}{-- nombre}{}
    \umlclass[x=-7,y=0]{Tenista}{-- gananciasAnteriores\\-- gananciasActuales\\-- partidosDavis}{}
    \umlclass[x=7,y=0]{Futbolista}{-- gananciasAcumuladas\\-- partidosSelección}{}
    \umlclass[x=7,y=-6]{Contrato}{-- añoInicio\\-- añoTérmino\\-- sueldoAnual}{}
    \umlclass[x=0,y=-6]{EquipoFútbol}{-- nombre}{}
    \umlassoc[name=estenista]{Tenista}{Deportista}
    \draw (estenista-1) {} node[above=2mm] (tr) {es $\blacktriangleright$};
    \umlassoc[name=esfutbolista]{Futbolista}{Deportista}
    \draw (esfutbolista-1) {} node[above=2mm] (tr) {$\blacktriangleleft$ es};
    \umlassoc[name=representa,attr1=|1,attr2=|*,pos1=0.1,pos2=0.9]{Manager}{Deportista}
    \draw (representa-1) {} node[above=2mm,rotate=45] (tr) {representa $\blacktriangleright$};
    \umlassoc[name=tiene,attr1=|1,attr2=|0..1,pos1=0.1,pos2=0.9]{Futbolista}{Contrato}
    \draw (tiene-1) {} node[above=2mm,rotate=-90] (tr) {tiene $\blacktriangleright$};
    \umlassoc[name=con,attr1=|*,attr2=|1,pos1=0.1,pos2=0.9]{Contrato}{EquipoFútbol}
    \draw (con-1) {} node[above=2mm] (tr) {$\blacktriangleleft$ con};
\end{tikzpicture}
\end{center}


\end{ProbPart}

\begin{ProbPart}

\begin{center}
\begin{tikzpicture}
    \umlclass[x=0,y=0,type=abstract]{Deportista}{-- nombre\\-- añoNac\\-- añosJugadorProf\\-- país}{\umlvirt{+ calculaGananciasTotales()}\\\umlvirt{+ haRepresentadoPaís()}}
    \umlclass[x=-6,y=-6]{Manager}{-- nombre}{+ obtListadoDeportistas()\\+ promSueldoAnualFutbolistas()\\+ promGananciasAñoAntTenistas()}
    \umlclass[x=-7,y=0]{Tenista}{-- gananciasAnteriores\\-- gananciasActuales\\-- partidosDavis}{+ calculaGananciasTotales()\\+ haRepresentadoPaís()}
    \umlclass[x=7,y=0]{Futbolista}{-- gananciasAcumuladas\\-- partidosSelección}{+ calculaGananciasTotales()\\+ haRepresentadoPaís()}
    \umlclass[x=2,y=-6]{EquipoFútbol}{-- nombre}{+ obtListadoJugadores()\\+ pagoTotalJugadores()}
    \umlinherit{Tenista}{Deportista}
    \umlinherit{Futbolista}{Deportista}
    \umlaggreg[name=enequipo,attr1=|0..1,attr2=|*,pos1=0.2,pos2=0.9]{EquipoFútbol}{Futbolista}
    \umlassocclass[x=7,y=-5]{Contrato}{enequipo-1}{-- añoInicio\\-- añoTérmino\\-- sueldoAnual}{}
    \umlassoc[attr1=representante|1,attr2=|*,pos1=0.2,pos2=0.9]{Manager}{Deportista}
    % \umlassoc[name=tiene,attr1=|1,attr2=|0..1,pos1=0.1,pos2=0.9]{Futbolista}{Contrato}
    % \umlassoc[name=con,attr1=|*,attr2=|1,pos1=0.1,pos2=0.9]{Contrato}{EquipoFútbol}
\end{tikzpicture}
\end{center}


\end{ProbPart}

\begin{ProbPart}

\begin{center}
\scalebox{0.8}{%
\begin{tikzpicture}
    \umlclass[x=0,y=0,type=abstract]{Deportista}{-- nombre\\-- añoNac\\-- añosJugadorProf\\-- país}{\umlvirt{+ calculaGananciasTotales()}\\\umlvirt{+ haRepresentadoPaís()}}
    \umlclass[x=-12,y=0]{Manager}{-- nombre}{+ obtListadoDeportistas()\\+ promSueldoAnualFutbolistas()\\+ promGananciasAñoAntTenistas()}
    \umlclass[x=-12,y=-6]{Tenista}{-- gananciasAnteriores\\-- gananciasActuales\\-- partidosDavis}{+ calculaGananciasTotales()\\+ haRepresentadoPaís()}
    \umlclass[x=0,y=-6]{Futbolista}{-- gananciasAcumuladas\\-- partidosSelección}{+ calculaGananciasTotales()\\+ haRepresentadoPaís()}
    \umlclass[x=3,y=-13]{EquipoFútbol}{-- nombre}{+ obtListadoJugadores()\\+ pagoTotalJugadores()}
    \umlclass[x=-3,y=-13]{Selección}{-- país}{}
    \umlclass[x=0,y=-17,type=abstract]{Plantel}{}{}
    \umlclass[x=-7,y=-17]{Entrenador}{-- gananciasAcumEntrenador}{}
%
    \umlVHVinherit[anchors=90 and -130]{Tenista}{Deportista}
    \umlinherit{Futbolista}{Deportista}
    \umlaggreg[name=enequipo,attr1=|0..1,attr2=|*,pos1=0.2,pos2=0.8]{EquipoFútbol}{Futbolista}
    \umlaggreg[attr1=|0..1,attr2=|*,pos1=0.2,pos2=0.8]{Selección}{Futbolista}
%
    \umlassoc[attr1=representante|1,attr2=|*,pos1=0.2,pos2=0.9]{Manager}{Deportista}
    \umlassoc[name=enseleccion,attr1=|1,attr2=|0..1,pos1=0.2,pos2=0.9]{Entrenador}{Plantel}
    \umlVHinherit{Entrenador}{Futbolista}
    \umlinherit{EquipoFútbol}{Plantel}
    \umlinherit{Selección}{Plantel}
%
    \umlassocclass[x=6,y=-9]{ContratoFubolista}{enequipo-1}{}{}
    \umlassocclass[x=-3,y=-20]{ContratoEntrenador}{enseleccion-1}{}{}
    \umlclass[x=6,y=-20,type=abstract]{Contrato}{-- añoInicio\\-- añoTérmino\\-- sueldoAnual}{}
    \umlinherit{ContratoFubolista}{Contrato}
    \umlinherit{ContratoEntrenador}{Contrato}
\end{tikzpicture}
}
\end{center}


\end{ProbPart}

\end{NewAnswer}

\end{spacing}
\end{document}
