## Ejemplo 1

Modelar la clase \bld{Préstamo}.\newline

Un préstamo tiene las siguientes características:

- Tasa de interés anual.
- Duración en años.
- Capital.
- Fecha de inicio.

Y tiene que ser capaz de lo siguiente:

- Proveer acceso a las características descritas arriba.
- Permitir la modificación de estas características.
- Obtener la cuota mensual.
- Obtener el costo total del crédito.

## Ejemplo 1

Modelar la clase \bld{Préstamo}.\newline

\centering\scalebox{0.5}{%
\begin{tikzpicture}
    \umlclass[x=0,y=0]{Préstamo}%
      {--tasaIntAnual\\ --periodoAnnos\\ --capital\\ --fechaInicio}%
      {%
      +Préstamo()\\
      +Préstamo(tasaIntAnual, periodoAnnos, capital)\\
      +obtTasaIntAnual()\\ +obtPeriodoAnnos()\\ +obtCapital()\\ +obtFechaInicio()\\
      +estTasaIntAnual(tasaIntAnual)\\ +estPeriodoAnnos(periodoAnnos)\\ +estCapital(capital)\\ +estFechaInicio(fechaInicio)\\
      +pagoMensual()\\ +costoTotal()
      }
\end{tikzpicture}
}

## Ejercicio 1

Elaborar un modelo básico de los clientes de un banco.

1. Un banco ofrece cuentas corrientes y cuentas vista a sus clientes.
1. Los clientes se relacionan con el banco a través de sus ejecutivos.
1. Debemos modelar todos los objetos involucrados en esta situación:
    - Propiedades de cada objeto.
    - Métodos de cada objeto.
1. No más de 5 tipos de objetos.\vfill


\begin{center}\begin{customRoundedBox}{}%
\begin{itemize}
  \item Tienen 15 minutos.
  \item Ilustren el modelo con un diagrama.
\end{itemize}
\end{customRoundedBox}
\end{center}

