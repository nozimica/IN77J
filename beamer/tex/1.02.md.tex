## Desarrollo Orientado a Objetos

El proceso completo de diseñar una solución orientada a objetos, para un
problema dado, está constituido por tres etapas consecutivas:\newline

\centering\begin{tikzflowchart}
  \node (a1) [concept, text width=40mm] {Análisis OO};
  \node (a2) [concept, text width=40mm, below=8mm of a1] {Diseño OO};
  \node (a3) [concept, text width=40mm, below=8mm of a2] {Implementación OO};
  \draw[->, thick] (a1) -- (a2);
  \draw[->, thick] (a2) -- (a3);
\end{tikzflowchart}
\vfill

## Análisis Orientado a Objetos

Es una manera de orientar el diseño ---o rediseño--- de la solución de un problema, el que puede ser de 
cualquier índole, por ejemplo un proceso.\newline

Consiste fundamentalmente en:

- Considerar tanto el dominio del problema como su solución lógica en términos de \bld{objetos}.
- Identificar los objetos relevantes para el problema.
- Describirlos mediante la especificación de sus \bld{propiedades} y \bld{acciones}.
- Considerar la manera en que interactúan entre sí.
- Concepto clave: abstracción.
    - Se modela un problema real desde un punto de vista abstracto.

## Análisis Orientado a Objetos

Resultado:

- Siempre obtendremos una primera aproximación a la solución del problema.
- La solución se compone de objetos con características propias.
    - Propias y relevantes para el problema abordado.
- Los objetos interactúan de una manera útil para la solución del problema.

## Diseño Orientado a Objetos

Se toma como punto de partida el resultado del análisis.\newline

A partir de los objetos ya identificados, se refina el modelo, considerando
limitaciones concretas tales como:

- \itt{Hardware} a utilizar.
- Lenguaje de programación escogido.
- Requerimientos de desempeño.
- Tipos de persistencia de datos (Bases de datos), etc.

## Diseño Orientado a Objetos

Resultado:

- Un diseño más cercano a la implementación final.
- Se define claramente los objetos que participarán en la solución.
    - Se establecerá las \bld{responsabilidades} de cada uno.
- Se podrá incluir objetos descartados en la fase de análisis y se
podrá descartar objetos anteriormente considerados.
- Se visualizará la eventual necesidad de \bld{objetos auxiliares}.

## Implementación Orientada a Objetos

Objetivos:

- Implementar finalmente el diseño en el lenguaje de programación escogido.
- Refinar el diseño para adecuarse estas eventualidades:
    - Modificaciones cuya necesidad de vislumbró recién en esta etapa.
    - Desajustes entre el diseño y determinados casos de uso definidos en
    los requerimientos.

Resultado:

- El \itt{software} que lleva a cabo la solución de nuestro problema o rediseño.

## Definiciones más formales de OOP

\subsubtitleB{Objetos:}

- Representa una entidad del mundo real.
- Esta entidad está claramente identificada.
    - Tiene límites bien definidos
    - Tiene una identidad única.

\begin{columns}[t,onlytextwidth]
\column{0.6\textwidth}
\begin{itemize}
\item Está compuesto por: 
    \begin{itemize}
        \item Un \tikz\node[inlineBlock] (propiedadesFrame) {\bld{estado}};.
        \item Un \tikz\node[inlineBlock] (comportamientoFrame) {\bld{comportamiento} definido};.
    \end{itemize}
\end{itemize}

\column{0.4\textwidth}
\begin{tikzpicture}[overlay]
  \node[miniconcept, right=50mm of propiedadesFrame] (propiedadesFig) { Atributos };
  \draw[->, thick] (propiedadesFig) -- ([xshift=3mm]propiedadesFrame.east);
  \node[miniconcept, below=6mm of propiedadesFig] (comportamientoFig) { Métodos };
  \draw[->, thick] (comportamientoFig) -- ([xshift=3mm]comportamientoFrame.east);
\end{tikzpicture}

\end{columns}

## Definiciones más formales de OOP

\subsubtitleB{Objetos:}\newline

Cada objeto es individual y tiene una \bld{identidad}.\newline

- Es una propiedad inherente de cada objeto el que pueda ser distinguible de los demás.
- Dos objetos distintos pueden compartir casi todas sus propiedades.
    - Dos gemelos son personas distintas.


## Definiciones más formales de OOP

\subsubtitleB{Clases:}

- Son las ``fábricas'' que construyen objetos de un mismo tipo.
- Corresponde más directamente a cada uno de los conceptos que se está modelando.
    - El objeto corresponde a una de las muchas \bld{instancias} de aquel concepto
    que estarán operando dentro de nuestra solución.

\centering\scalebox{0.7}{%
\begin{tikzpicture}[font={\baselineskip=2pt}]
    \umlemptyclass[x=0,y=0]{Crédito}{}{}
    \umlemptyclass[x=3,y=0]{Persona}{}{}
    \umlemptyclass[x=6,y=0]{Aval}{}{}
\end{tikzpicture}
}

## Definiciones más formales de OOP

\subsubtitleB{Propiedades o atributos:}

- Son las características relevantes de cada objeto de una clase.
- Tienen un nombre que las identifica dentro de la clase.
- Pueden tener un tipo de datos asociado.
- La clase puede limitar el acceso de estas propiedades desde el ``mundo exterior''.

\centering\scalebox{0.7}{%
\begin{tikzpicture}
    \umlclass[x=0,y=0]{Persona}%
      {--Nombre\\ --Edad\\ --IdiomaNativo\\ --Residencia}%
      {}
    \umlclass[x=4,y=0]{Camión}%
      {--Patente\\ --Año\\ --MáxCarga}%
      {}
    \umlclass[x=8,y=0]{Zorzal}%
      {--ID\\ --Edad\\ --Género\\ --Hábitat}%
      {}
\end{tikzpicture}
}

## Definiciones más formales de OOP

\subsubtitleB{Propiedades o atributos:}\newline

Pueden ser:

\begin{description}[leftmargin=3em]
    \item[De instancia:] Cada objeto de la clase tiene un valor independiente.
    \item[De clase:] La propiedad le pertenece a la clase: todos los objetos
    comparten el mismo valor.
\end{description}

\vfill
\bld{Estado de un objeto:}

- Es el conjunto de las propiedades de un objeto en un momento dado.

## Definiciones más formales de OOP

\subsubtitleB{Métodos:}

- Modelan el comportamiento de una clase de objetos.
- Se consideran las acciones relevantes para el problema.
- Invocar un método es \itt{ordenarle} a un objeto que haga una tarea.
- Existen métodos especiales: \bld{constructores}.

\vfill
Son muy similares a las \bld{funciones} de la programación procedural:

- Pueden recibir parámetros.
- Pueden entregar un resultado tras ser invocados.

## Definiciones más formales de OOP

\subsubtitleB{Métodos:}

\vfill

\centering\scalebox{0.7}{%
\begin{tikzpicture}
    \umlclass[x=0,y=0]{Persona}%
      {--Nombre\\ --Edad\\ --IdiomaNativo\\ --Residencia}%
      {+Habla()\\ --Respira()\\ +DiceNombre()\\ +Camina()\\ --Sueña()}
    \umlclass[x=4,y=0]{Camión}%
      {--Patente\\ --Año\\ --MáxCarga}%
      {+encender()\\ --revisaComb()\\ +prendeLuces()}
    \umlclass[x=8,y=0]{Zorzal}%
      {--ID\\ --Edad\\ --Género\\ --Hábitat}%
      {+nace() \\ +vuela() \\ --come() \\ +muere()}
\end{tikzpicture}
}

## Definiciones más formales de OOP

\subsubtitleB{Visibilidad de propiedades y métodos:}\newline

¿Se fijaron en los signos \bld{+} y \bld{--} en los Diagramas de Clase anteriores?\newline

Existen fundamentalmente tres tipos de visibilidad:

\begin{description}[leftmargin=3em]
    \item[Pública (+):] Accesible y visible para todos los demás objetos.
    \item[Privada (--):] Accesible y visible sólo dentro del propio objeto.
    \item[Protegida (\#):] Accesible y visible sólo para objetos de la misma clase o de sus herederos.
\end{description}

## Definiciones más formales de OOP

\subsubtitleB{Visibilidad de propiedades y métodos:}\newline

Recomendaciones:

- Que las propiedades sean privadas.
    - Para acceder o modificar propidedades: métodos de acceso y de modificación.
- Que los métodos sean públicos.

## Definiciones más formales de OOP

\subsubtitleB{Abstracción de Clase y Encapsulación:}\newline

La \bld{abstracción de clase} es:

- La separación entre la implementación de la clase y la manera en que ésta será
finalmente usada.
- Quien crea una clase provee una descripción de la clase y sus métodos que se pueden
usar.
    - Esto es el ``contrato de la clase''.


La \bld{encapsulación} es:

- El hecho que los detalles de la implementación están ocultos para quien usa la clase.

\vfill

Estos dos conceptos son las dos caras de una misma moneda.

## Definiciones más formales de OOP

\subsubtitleB{Herencia}

- Es la derivación de nuevas clases a partir de otras ya existentes.
- Una clase heredera es una versión más \itt{específica} que la heredada.
- Este es uno de los conceptos más importantes en OOP.


\centering\scalebox{0.7}{%
\begin{tikzpicture}
    \umlclass[x=0,y=0]{Persona}{}{}
    \umlclass[x=-2,y=-3]{Cliente}{}{}
    \umlclass[x=2,y=-3]{Aval}{}{}
    \umlinherit{Cliente}{Persona}
    \umlinherit{Aval}{Persona}
    % \umlrelation[style={tikzuml inherit style}]{Cliente}{Persona}
\end{tikzpicture}
}

\begin{center}
% ## Definiciones más formales de OOP

% Polimorfismo: diferentes clases pueden verse a través de interfaces comunes

% ## Definiciones más formales de OOP

% \subsubtitleB{Objetos, propiedades y métodos:}

% - Encapsulan tanto su \bld{estado} como su \bld{comportamiento}.

\end{center}
