## Diagrama de Secuencia

### Dados: diagrama de Secuencia

\subsubtitleB{Diagrama de Secuencia}

Ya tenemos identificados el dominio, las clases y los casos de uso.\newline

\pause

Nos está faltando algo muy importante:

\begin{rboxx}{}
La interacción entre las clases
\end{rboxx}

\pause

Para cubrir la interacción entre las componentes de nuestro sistema, es que UML
incluye un nuevo diagrama:

\begin{rboxx}{}
\texthigh{Diagrama de secuencia}
\end{rboxx}


### Dados: diagrama de Secuencia

\subsubtitleB{Diagrama de Secuencia}

- Un diagrama de secuencia describe gráficamente los pasos que ya hemos visualizado en la narrativa
del caso de uso.

- Complementa esto último con una visualización gráfica de los actores involucrados en estos pasos.

- Nos permite entender mucho mejor los métodos o acciones que los objetos deben proveer.

### Dados: diagrama de Secuencia

\subsubtitleB{Diagrama de Secuencia}

\tikzinlinec[0.5]{
    \begin{umlseqdiag}
    \umlactor[no ddots]{Jugador}
    \umlbasicobject{Pantalla}
    \umlbasicobject{Juego}
    \umlobject[class=Dado]{d1}
    \umlobject[class=Dado]{d2}
    \begin{umlcall}[op=jugar()]{Jugador}{Juego}
    \end{umlcall}
    \begin{umlcall}[op={muestra(``Inicia el juego'')}]{Juego}{Pantalla}
    \end{umlcall}
    \begin{umlcall}[op=tirar()]{Juego}{d1}
    \end{umlcall}
    \begin{umlcall}[op=pideValor(),return=v1]{Juego}{d1}
    \end{umlcall}
    \begin{umlcall}[op=muestra(v1)]{Juego}{Pantalla}
    \end{umlcall}
    \begin{umlcall}[op=tirar()]{Juego}{d2}
    \end{umlcall}
    \begin{umlcall}[op=pideValor(),return=v2]{Juego}{d2}
    \end{umlcall}
    \begin{umlcall}[op=muestra(v2)]{Juego}{Pantalla}
    \end{umlcall}
    \begin{umlcall}[op=muestra(``Termina el juego'')]{Juego}{Pantalla}
    \end{umlcall}
    \end{umlseqdiag}
}

### Diagrama de Secuencia

\subsubtitleB{En qué consisten:}

- Muestran la secuencia de mensajes entre los objetos que interactúan en un escenario en particular.
- Están compuestos de dos dimensiones:
    - Horizontal: los objetos se disponen de izquierda a derecha.
    - Vertical: el tiempo avanza desde arriba hacia abajo.
        - El primer mensaje es el que está más arriba.

\tikzinlinec[0.6]{
    \begin{umlseqdiag}
    \umlobject[class=Clase1]{obj1}
    \umlobject[class=Clase2]{obj2}
    \umlobject[class=Clase3]{obj3}
    \begin{umlcall}[op=primeraOp()]{obj1}{obj2}
    \end{umlcall}
    \begin{umlcall}[op=segundaOp()]{obj1}{obj3}
    \end{umlcall}
    \end{umlseqdiag}
}

### Diagrama de Secuencia

\subsubtitleB{Qué modelan:}

- Modelan las interacciones a alto nivel:
    - Entre los objetos activos de un sistema.
    - Entre un usuario y el sistema.
    - Entre varios sistemas.
- Dado un caso de uso:
    - Modelan todos los pasos de un escenario, ilustrando la interaccion entre los
    objetos involucrados.
- Son capaces de modelar determinados detalles de la interacción, tales como ciclos
o decisiones.

## Diagrama de Secuencia: Elementos

### Diagrama de Secuencia

\subsubtitleB{Elementos: objetos}

- Los objetos son los elementos dispuestos desde izquierda a derecha.
- Una línea punteada avanzando hacia abajo indica su línea de vida:
    - Dentro de ella se indicará con un rectángulo blanco el momento dentro de la
    secuencia en que este objeto está activo.
- Puede corresponder a:
    - Una instancia específica de una clase.
    - Una instancia cualquiera de una clase.
    - Una clase en particular.
    - Un actor externo al sistema.
    - Una base de datos.

### Diagrama de Secuencia

\subsubtitleB{Elementos: tipos de objetos}

- Corresponden a una de las siguientes alternativas:
    - Instancia de una clase en particular:
\vspace{0.5em}
\tikzinlinec[0.7]{
    \begin{umlseqdiag}
    \umlobject[class=nombreClase]{nombreInstancia}
    \end{umlseqdiag}
}
\vspace{0.5em}
    - Un actor externo al sistema:
\tikzinlinec[0.7]{
    \begin{umlseqdiag}
    \umlactor[no ddots]{nombreActor}
    \end{umlseqdiag}
}
\vspace{0.5em}
    - Una base de datos:
\tikzinlinec[0.7]{
    \begin{umlseqdiag}
    \umldatabase[no ddots]{nombreBaseDeDatos}
    \end{umlseqdiag}
}

### Diagrama de Secuencia

\subsubtitleB{Elementos: tipos de objetos}

- También se puede diferenciar los objetos en según su función en el sistema:
    - Una instancia que sirve de \bld{frontera} (o interfaz) entre el sistema y sus actores.
\tikzinlinec[0.7]{
    \begin{umlseqdiag}
    \umlboundary[class=nombreClase]{nombreInstancia}
    \end{umlseqdiag}
}
    - Esta es una instancia de una clase como cualquier otra, salvo por un poderoso detalle:
    sirve de vía de comunicación entre el sistema y el mundo exterior:
        - Puede representar toda la interfaz gráfica del sistema.
        - Puede corresponder a una consola de texto.
        - Es lo que en la arquitectura MVC se denominta \bld{vista}.

### Diagrama de Secuencia

\subsubtitleB{Elementos: tipos de objetos}

- También se puede diferenciar los objetos en según su función en el sistema:
    - Una instancia que representa las \bld{entidades}, que son los datos o información manejados por el sistema.
\tikzinlinec[0.7]{
    \begin{umlseqdiag}
    \umlentity[class=nombreClase]{nombreInstancia}
    \end{umlseqdiag}
}
    - No es la base de datos propiamente tal, sino que una clase que se encargará de la
    \bld{persistencia} de los datos.
        - Esta persistencia puede ser lograda tanto con bases de datos como con otros medios.
        - Es lo que en la arquitectura MVC se denominta \bld{modelo}.

### Diagrama de Secuencia

\subsubtitleB{Elementos: tipos de objetos}

- También se puede diferenciar los objetos en según su función en el sistema:
    - Una instancia que ejerce el \bld{control} entre la frontera y las entidades.
\tikzinlinec[0.7]{
    \begin{umlseqdiag}
    \umlcontrol[class=nombreClase]{nombreInstancia}
    \end{umlseqdiag}
}
    - La frontera nunca se comunica directamente con las entidades.
        - Para esta comunicación siempre estará entre medio el controlador.
        - Es lo que en la arquitectura MVC se denominta \bld{controlador}.

### Diagrama de Secuencia

\subsubtitleB{Elementos: mensajes}

- Los \bld{mensajes} entre dos objetos se representan mediante \bld{flechas} que unen las líneas de
vida de esos dos objetos.
    - Lo normal es que la ``entrega'' del mensaje sea inmediata:
\vspace{0.5em}
\tikzinlinec[0.7]{
    \begin{umlseqdiag}
    \umlbasicobject{Cajero}
    \umlbasicobject{Caja}
    \begin{umlcall}[op=crearVenta(),padding=3]{Cajero}{Caja}
    \end{umlcall}
    \end{umlseqdiag}
}
    - Si se sabe que habrá una \bld{demora}, la línea se dibuja \bld{oblicua hacia abajo}:
\vspace{0.5em}
\tikzinlinec[0.7]{
    \begin{umlseqdiag}
    \umlobject[class=Cliente]{cliObj}
    \umlobject[x=6,class=ServTempHoraria]{tempHorObj}
    \begin{umlcall}[delay=true,op=pideTemperatura(),return=temperatura,padding=3]{cliObj}{tempHorObj}
    \end{umlcall}
    \end{umlseqdiag}
}

### Diagrama de Secuencia

\subsubtitleB{Elementos: mensajes}

- Cada flecha tiene escrita sobre ella el \bld{nombre} del mensaje correspondiente.
    - Este mensaje puede ser la invocación de un método sobre el objeto receptor.
    - En estos casos se puede especificar también los argumentos entregados a ese método.

### Diagrama de Secuencia

\subsubtitleB{Elementos: mensajes}

- Por \bld{cada mensaje}, se generan sendos \bld{rectángulos} en las líneas de vida de los objetos involucrados.
    - Estos \bld{rectángulos} indican el \bld{tiempo} que se toma el mensaje en estar activo.
- De ser necesario, se puede indicar el \bld{valor retornado} por el mensaje.
    - Se hace mediante una \bld{línea discontínua (guiones)}.
    - Puede indicar el nombre del dato retornado.

\tikzinlinec[0.7]{
    \begin{umlseqdiag}
    \umlobject[class=Catálogo]{catObj}
    \umlobject[x=6,class=Producto]{prod1}
    \begin{umlcall}[op=pidePrecio(),return=precio,padding=3]{catObj}{prod1}
    \end{umlcall}
    \end{umlseqdiag}
}

### Diagrama de Secuencia

\subsubtitleB{Elementos: mensajes}

- Los mensajes pueden ser:
\vfill
\begin{description}
    \item[Síncronos:] si el \bld{objeto origen} del mensaje \bld{debe esperar} a que \bld{termine} el llamado antes
    de continuar.
        \begin{itemize}
            \item Típicamente corresponde a la invocación de un método.
        \end{itemize}
\vfill
    \item[Asíncronos:] si el \bld{objeto origen no se queda esperando} por el término del mensaje.
        \begin{itemize}
            \item Permite que los objetos interactúen concurrentemente.
            \item De todas maneras el objeto destino puede retornar un valor al objeto origen.
            \item Es una flecha con cabezal no relleno.
        \end{itemize}
\end{description}

### Diagrama de Secuencia

\subsubtitleB{Elementos: mensajes}

\bld{Ejemplo:}

- Se tiene un estudiante que consulta una página con su información académica.
- Después de ser creada, el estudiante la imprime.
    - El objeto que representa a esa página envía a su vez la orden a la impresora física.
    - En esa orden, la misma página es el parámetro.
    - Este llamado u orden es asíncrono, pues no hay necesidad de esperar a que estén las hojas impresas
    para que el estudiante pueda seguir navegando el sistema.

### Diagrama de Secuencia

\subsubtitleB{Elementos: mensajes}

\bld{Ejemplo:}

\tikzinlinec[0.6]{
    \begin{umlseqdiag}
    \umlactor[no ddots]{Estudiante}
    \umlcreatecall[stereo=boundary,class=PagInfoEstudiante,x=6]{Estudiante}{estudPag}
    \umlactor[x=12,no ddots]{Impresora}
    \begin{umlcall}[op=imprimeBoletin()]{Estudiante}{estudPag}
        \begin{umlcall}[type=asynchron,op=imprimePagina(estudPag)]{estudPag}{Impresora}
        \end{umlcall}
    \end{umlcall}
    \end{umlseqdiag}
}


### Diagrama de Secuencia

\subsubtitleB{Elementos: mensajes a sí mismo}

- Hay veces que un objeto debe enviarse un mensaje a sí mismo.
    - Un objeto puede proveer varios métodos, donde algunos sirven como ``auxiliares'' de otros.
    - Por ejemplo, un objeto GUI puede mostrar un formulario de ingreso de datos que está
    oculto mientras el usuario no lo pida.
    - O, por ejemplo, un objeto debe actualizar la fecha de la última modificación de sí mismo.

### Diagrama de Secuencia

\subsubtitleB{Elementos: mensajes a sí mismo}

\tikzinlinec[0.5]{
    \begin{umlseqdiag}
    \umlactor[no ddots]{Usuario}
    \umlboundary[class=ContactosGUI]{g}
    \umlcontrol[class=ContactosControl]{c}
    \umlentity[class=Contactos]{contactos}
    \umlentity[class=Persona]{pers}
    \begin{umlcall}[op=agregarNuevo(),padding=-1]{Usuario}{g}
        \begin{umlcall}[op=muestraFormAgregar()]{g}{g}
        \end{umlcall}
    \end{umlcall}
    \begin{umlcall}[op=envíaForm(),padding=-1]{Usuario}{g}
        \begin{umlcall}[op=ejecutaEnvío(data)]{g}{c}
            \begin{umlcall}[op=agregaContacto(data),return=éxito,padding=-2]{c}{contactos}
                \begin{umlcall}[op=creaNvo(data)]{contactos}{pers}
                \end{umlcall}
                \begin{umlcallself}[op=fechaUltModif()]{contactos}
                \end{umlcallself}
            \end{umlcall}
        \end{umlcall}
    \end{umlcall}
    \end{umlseqdiag}
}

### Diagrama de Secuencia

\subsubtitleB{Elementos: bloques}

- Los diagramas de secuencia permiten encerrar algunos mensajes dentro de \bld{bloques}.
- Estos bloques permiten:
    - Establecer condiciones para la ejecución de los mensajes dentro de él.
        \begin{rboxx}{}
        Bloques condicionales.
        \end{rboxx}
    - Repetir todos los mensajes dentro del bloque mientras se dé alguna condición.
        \begin{rboxx}{}
        Bloques iterativos.
        \end{rboxx}

### Diagrama de Secuencia

\subsubtitleB{Elementos: bloques}

- \bld{Cuidado}: si se quiere modelar el flujo de un programa en mayor detalle, un diagrama de secuencia
puede no ser la mejor alternativa.
    - Para ello, también podemos usar:

\buildrboxx{}

- Diagramas de actividades.
- Diagramas de flujo.
- Pseudocódigo.
- Código propiamente tal.

\finishrboxx

- De todas maneras es conveniente saber cómo usar bloques: pueden ser de mucha ayuda en determinados casos.

### Diagrama de Secuencia

\subsubtitleB{Elementos: bloques}

- Un \bld{bloque} es un \bld{rectángulo}:
    - Con una etiqueta en la esquina superior izquierda, que indica el tipo de bloque.
    - Opcionalmente, con un texto encerrado entre corchetes, llamado condición de guarda, que establece la condición del bloque.
    - Opcionalmente, puede estar subdividido en varias secciones, mediante una línea horizontal segmentada.

### Diagrama de Secuencia

\subsubtitleB{Elementos: bloques}

- Por ejemplo:

\tikzinlinec[0.5]{
    \begin{umlseqdiag}
    \umlactor[no ddots]{Cliente}
    \umlbasicobject{Orden}
    \umlbasicobject{EnvioNormal}
    \umlbasicobject{EnvioEspecial}
    \begin{umlcall}[op=finalizaOrden()]{Cliente}{Orden}
    \end{umlcall}
    \begin{umlcall}[op=calculaValor(),return=valor]{Orden}{Orden}
    \end{umlcall}
    \begin{umlfragment}[type=alt,label=valor>10000,inner xsep=10]
        \begin{umlcall}[op=despacha(),dt=7]{Orden}{EnvioEspecial}
        \end{umlcall}
        \umlfpart[si no]
        \begin{umlcall}[op=despacha()]{Orden}{EnvioNormal}
        \end{umlcall}
    \end{umlfragment}
    \end{umlseqdiag}
}

### Diagrama de Secuencia

\subsubtitleB{Elementos: bloques condicionales}

- Una de las ventajas de un diagrama de secuencia es que permite incorporar ---sólo en caso
de ser necesario--- bloques condicionales.
    - Permite especificar cómo se comporta la interacción entre objetos \bld{con mayor detalle}.
    - Incorporar bloques condicionales \bld{acerca} este diagrama al área de la \bld{implementación}.
        - Usar esto con precaución.

### Diagrama de Secuencia

\subsubtitleB{Elementos: bloques condicionales}

- Fundamentalmente hay dos tipos de bloques condicionales:

\begin{description}
    \item[alt]: Un bloque con varios fragmentos, de los cuales solamente se activará aquel cuya
    condición sea verdadera.
    \item[opt:] Es una sola condición, sin otras alternativas.
\end{description}

### Diagrama de Secuencia

\subsubtitleB{Elementos: bloques condicionales}

\bld{Ejemplo:}\newline

- En el diagrama donde un usuario ingresaba un nuevo contacto, ocurría algo especial:
    - Tras agregar un contacto, se retornaba el éxito de esa operación.
    - Este valor retornado llegaba al objeto que invocó ese método, que era el \bld{controlador}.

\vfill

- ¿Era suficiente con que el controlador se enterase del resultado de ingresar un contacto?

\pause

\buildrboxx{}

- \bld{No} era suficiente: el usuario fue quien inició el proceso y esperará algún \itt{feedback}.

\finishrboxx

### Diagrama de Secuencia

\subsubtitleB{Elementos: bloques condicionales}

\tikzinlinec[0.5]{
    \begin{umlseqdiag}
    \umlactor[no ddots]{Usuario}
    \umlboundary[class=ContactosGUI]{g}
    \umlcontrol[class=ContactosControl]{c}
    \umlentity[class=Contactos]{contactos}
    \umlentity[class=Persona]{pers}
    \begin{umlcall}[op=agregarNuevo(),padding=-1]{Usuario}{g}
        \begin{umlcall}[op=muestraFormAgregar()]{g}{g}
        \end{umlcall}
    \end{umlcall}
    \begin{umlcall}[op=envíaForm(),padding=-1]{Usuario}{g}
        \begin{umlcall}[op=ejecutaEnvío(data)]{g}{c}
            \begin{umlcall}[op=agregaContacto(data),return=éxito,padding=-2]{c}{contactos}
                \begin{umlcall}[op=creaNvo(data)]{contactos}{pers}
                \end{umlcall}
                \begin{umlcallself}[op=fechaUltModif()]{contactos}
                \end{umlcallself}
            \end{umlcall}
            \begin{umlfragment}[type=alt,label=éxito,inner xsep=5]
                \begin{umlcall}[op=avisaÉxito()]{c}{g}
                \end{umlcall}
                \umlfpart[si no]
                \begin{umlcall}[op=avisaError()]{c}{g}
                \end{umlcall}
            \end{umlfragment}
        \end{umlcall}
    \end{umlcall}
    \end{umlseqdiag}
}

### Diagrama de Secuencia

\subsubtitleB{Elementos: bloques condicionales}

- En el ejemplo anterior, podemos darnos cuenta de varias cosas:

    - Es un bloque condicional de tipo \texthigh{alt}.
    - Que cuenta con dos fragmentos.
    - El primero se activa si el valor de ``éxito'' es verdadero.
    - El segundo en caso contrario.
    - En ambos casos, el controlador ejecutará un método ad-hoc de la GUI:
        - Ambos servirán para informarle al usuario del resultado de la operación.
        - A la GUI le basta con ejecutar esos métodos, no tiene que invocar
        ningún ``método'' sobre el usuario.

### Diagrama de Secuencia

\subsubtitleB{Elementos: bloques condicionales}

\bld{Ejemplo: pedir un libro en una biblioteca}\newline

- Una persona puede reservar un libro de una biblioteca.
- De estar disponible el libro, lo puede ir a buscar.
- Tras leerlo, lo devuelve a la biblioteca.
- La biblioteca, al momento de recibir la devolución de un libro, cobrará una multa al usuario
en caso de estar atrasado.

### Diagrama de Secuencia

\subsubtitleB{Elementos: bloques condicionales}

\bld{Ejemplo: pedir un libro en una biblioteca}\newline

\vspace{-1em}
\tikzinlinec[0.4]{
    \begin{umlseqdiag}
    \umlactor[no ddots]{Lector}
    \umlboundary[no ddots]{PagWeb}
    \umlcontrol[no ddots]{Controlador}
    \umlentity[no ddots]{Catalogo}
    \umlbasicobject{Biblioteca}
    \begin{umlcall}[op=consultar(titulo),return=disponible]{Lector}{PagWeb}
        \begin{umlcall}[op=consultar(titulo),return=disponible]{PagWeb}{Controlador}
            \begin{umlcall}[op=consultar(titulo),return=disponible]{Controlador}{Catalogo}
            \end{umlcall}
        \end{umlcall}
    \end{umlcall}
    \begin{umlfragment}[type=opt,label=disponible,inner xsep=7]
        \begin{umlcall}[op=pedir(titulo),return=libObj,dt=7]{Lector}{Biblioteca}
        \umlcreatecall[class=Libro,x=6]{Biblioteca}{libObj}
        \end{umlcall}
        \begin{umlcall}[op=abrir()]{Lector}{libObj}
        \end{umlcall}
        \begin{umlcall}[op=devolver(libObj)]{Lector}{Biblioteca}
            \begin{umlcall}[op=fechaLim(libObj),return=fecha]{Biblioteca}{Catalogo}
            \end{umlcall}
            \begin{umlfragment}[type=opt,label=fecha < hoy,inner xsep=7]
                \begin{umlcall}[op=multar()]{Biblioteca}{Lector}
                \end{umlcall}
            \end{umlfragment}
        \end{umlcall}
    \end{umlfragment}
    \end{umlseqdiag}
}

### Diagrama de Secuencia

\subsubtitleB{Elementos: bloques iterativos}

- En este caso, los mensajes dentro del bloque se repetirán \bld{mientras} se cumpla la condición definida en
la condición de guarda.

\bld{Ejemplo: reservar hotel por varios días}\newline

\vspace{-1em}
\tikzinlinec[0.5]{
    \begin{umlseqdiag}
    \umlactor[no ddots]{Cliente}
    \umlbasicobject{Hotel}
    \umldatabase{DbHabitaciones}
    \begin{umlcall}[op=reservar(fechas),return=resObj]{Cliente}{Hotel}
        \begin{umlfragment}[type=loop,label=para cada día,inner xsep=7]
            \begin{umlcall}[op=hayDisp(día),return=disp]{Hotel}{DbHabitaciones}
            \end{umlcall}
        \end{umlfragment}
        \begin{umlfragment}[type=opt,label=todas disp,inner xsep=7]
            \umlcreatecall[class=Reserva]{Hotel}{resObj}
        \end{umlfragment}
    \end{umlcall}
    \end{umlseqdiag}
}

### Diagrama de Secuencia de Sistema

\subsubtitleB{Definición}

- Para sistemas \texthigh{grandes}, es bueno pensar primero solamente en su interacción con sus usuarios.
    - Todo lo que sucede dentro del sistema es desconocido.
    - El sistema es una \texthigh{caja negra}.
- Se trabaja a \texthigh{partir del diagrama de casos de uso principal}.
    - Todo el sistema es una unidad.
    - Se especifica concretamente cómo interactúa el sistema con sus usuarios en cada caso de uso.

### Diagrama de Secuencia de Sistema

\subsubtitleB{Definición}

- Nuevamente nos enfocamos en \texthigh{qué} es lo que se espera que haga el sistema.
- Se toma en cuenta las interacciones de los usuarios sobre el sistema.
    - No en sentido inverso.
- Puede poner énfasis en los parámetros de los mensajes.

### Diagrama de Secuencia de Sistema

\subsubtitleB{Eventos}

- Son estímulos externos.
- Son generados por los actores.
- Ante ellos, el sistema \bld{debe reaccionar}.

\pause

\vfill

¿Cómo identificarlos?

- Las \texthigh{acciones} de los actores sobre el sistema dan una \texthigh{pista} de los eventos.
- Necesariamente tienen que ser eventos externos c/r al sistema.
- Dependen de dónde está trazada la línea divisoria entre el sistema y el mundo exterior.

### Diagrama de Secuencia de Sistema

\subsubtitleB{Eventos}

- Un cliente visita la librería y mira libros.
    - No es un evento, pues este hecho no afecta al sistema.

\vfill

- Un cliente llega con un libro a la caja para cancelarlo.
    - Tampoco es un evento, pues es una interacción entre actores.


\vfill

- El cajero ingresa la identificación del libro.
    - Sí es un evento, pues el sistema debe reaccionar ante esto entregando, por ejemplo,
    el valor del libro.

### Diagrama de Secuencia de Sistema

\subsubtitleB{Operaciones del sistema}

- Los eventos, al ser activados, disparan operaciones sobre el sistema.

\vfill

- El sistema debe estar preparado para ejecutar esas opciones ante cada evento.

\vfill

- Es la ``instancia sistema'' la encargada de ejecutar estas operaciones.
    - Esta es una instancia casi siempre ficticia, pues en la fase de diseño
    se establecerá las clases que realmente participarán en la interacción.

### Diagrama de Secuencia de Sistema

\subsubtitleB{Operaciones del sistema}

Si consideramos un \texthigh{Cajero} que interactúa con un \texthigh{Cliente} para procesar su compra,
tenemos:

\vfill

- ``El cajero ingresa el identificador de un producto en el sistema'':
    - Por ejemplo, leyendo su código de barras.
    - Esto genera una operación \texttt{agregarProducto(producto, cantidad)}.


\vfill

- ``El cajero finaliza la venta'':
    - Para que el sistema lleve a cabo los procesos finales.
    - Esto genera una operación \texttt{terminarVenta()}.

### Diagrama de Secuencia de Sistema

\subsubtitleB{Ejemplo:}

\tikzinlinec[0.5]{
    \begin{umlseqdiag}
    \umlactor[no ddots,x=0]{Cajero}
    \umlbasicobject[x=8]{Sistema}
    \begin{umlcall}[op=iniciarVenta()]{Cajero}{Sistema}
    \end{umlcall}
    \begin{umlfragment}[type=loop,label=para cada producto,inner xsep=13]
        \begin{umlcall}[op=agregarProd({producto,cantidad}),dt=7,return=subtotal]{Cajero}{Sistema}
        \end{umlcall}
    \end{umlfragment}
    \begin{umlcall}[op=terminarVenta(),return=total,dt=7]{Cajero}{Sistema}
    \end{umlcall}
    \begin{umlcall}[op=realizaPago(total),return={cambio,boleta},dt=7]{Cajero}{Sistema}
    \end{umlcall}
    \end{umlseqdiag}
}

### Ejercicio en clases

\subsubtitleB{El sistema del bibliometro:}

Analicemos el caso del bibliometro, pensando en los siguientes requisitos:

- Gestionar los libros y revistas.
    - Características.
    - Disponibilidad.
    - Administración.
- Gestionar los socios.
    - Datos.
    - Préstamos.
    - Administración.
- Gestionar los préstamos.
