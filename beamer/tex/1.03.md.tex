# Relaciones entre Clases

## Asociación

### Relaciones entre Clases

\subsubtitleB{Asociación}\newline

Hasta ahora hemos visto atributos que son simplemente datos:

- Edad
- Tipo de cliente
- Velocidad de procesador

Imaginémonos qué pasa al modelar una orden de compra.

### Relaciones entre Clases

\subsubtitleB{Asociación}\newline

Pero qué pasa si estamos modelando la siguiente situación:

- Existen \textcolor{blue}{Órdenes de compra}.

    \begin{itemize}
        \item Asociadas a una fecha.
        \item Si fue prepagada.
        \item Un \tikz\node[inlineBlock] (clienteFrame) {Cliente que la creó.};
    \end{itemize}

- Existen \textcolor{blue}{Clientes}.
    - Tienen un nombre.
    - Una dirección.
    - Una antigüedad.

\pause

\begin{tikzpicture}[overlay]
  \node[miniconcept, right=20mm of clienteFrame] (clienteFig) { Instancia de Cliente };
  \draw[->, thick] (clienteFig.west) -- ([xshift=3mm]clienteFrame.east);
\end{tikzpicture}

\pause

\begin{tikzpicture}[overlay]
\tiny
  \node[rectangle, below=15mm of clienteFig, text width=10em, align=center] (clienteComment) {Un cliente crea\\ninguna, una o varias\\órdenes de compra.};
  \draw[dotted] (clienteFig) -- (clienteComment);
\end{tikzpicture}

### Relaciones entre Clases

\subsubtitleB{Asociación}\newline

La asociación entre clases es un concepto que permite conectar clases que están relacionadas.\newline

Si bien son un atributo más en una clase, en un diagrama UML no se escribe en el cuadro de los
atributos:

\vfill

\begin{center}
\scalebox{0.5}{%
\begin{tikzpicture}
    \umlclass[x=0,y=0]{Orden}{--fecha : Integer\\--prepagada : Boolean}{}
    \umlclass[x=6,y=0]{Cliente}{--nombre : String\\--dirección : String\\--antigüedad : Integer}{}
    \umluniassoc{Orden}{Cliente}
\end{tikzpicture}
}
\end{center}

La flecha indica el sentido de la asociación.

### Relaciones entre Clases

\subsubtitleB{Asociación}\newline

Pero a veces no basta con dibujar una flecha, porque puede ser necesario:

- Indicar la multiplicidad:
    - Cuántas instancias de una clase pueden estar asociadas con otra.
- Darle un nombre a la asociación:
    - Es el nombre del rol.

\vfill
\begin{center}
\scalebox{0.5}{%
\begin{tikzpicture}
    \umlclass[x=0,y=0]{Orden}{--fecha : Integer\\--prepagada : Boolean}{}
    \umlclass[x=10,y=0]{Cliente}{--nombre : String\\--dirección : String\\--antigüedad : Integer}{}
    \umluniassoc[mult1=*,pos1=0.1,mult2=1,arg2=Emisor,pos2=0.85]{Orden}{Cliente}
\end{tikzpicture}
}
\end{center}


## Herencia

### Relaciones entre Clases

\subsubtitleB{Taxonomía}\newline

Cuando las Clases que componen un modelo tienen relaciones taxonómicas entre ellas, se
dice que existe una relación de generalización o de especialización.\newline

\bld{Generalización}\newline

De varias clases se identifican elementos comunes, los que dada su importancia será
más conveniente agruparlos en una clase nueva, relacionada a las clases originales.

\vspace{-1em}
\begin{center}\hspace{60mm}\begin{customRoundedBox}{}
\centering Se pasa de lo \bld{específico} a lo \bld{genérico}.
\end{customRoundedBox}\end{center}

### Relaciones entre Clases

\subsubtitleB{Taxonomía}\newline

\bld{Especialización}\newline

El elemento más específico consta de todas las propiedades y métodos del más
genérico y puede contener información adicional.

\vspace{-1em}
\begin{center}\hspace{60mm}\begin{customRoundedBox}{}
\centering Se pasa de lo \bld{genérico} a lo \bld{específico}.
\end{customRoundedBox}\end{center}


### Relaciones entre Clases

\subsubtitleB{Taxonomía}\newline

Un típico ejemplo de cómo se va desde lo general a lo específico son las taxonomías:\newline

\taxonomyOrg


### Relaciones entre Clases

\subsubtitleB{Clase Base y Clase Derivada}\newline

Al haber una relación de generalización entre dos Clases, se tiene que:

- A la clase más genérica se le denomina:
    - Clase Base.
    - Clase Padre.
    - Superclase.

- A la clase más específica se le denomina:
    - Clase Derivada.
    - Clase Hijo.
    - Subclase.

### Relaciones entre Clases

\subsubtitleB{Herencia}\newline

La manera de representar la generalización de clases en OOP se denomina ``Herencia'':

- Se derivan nuevas clases a partir de otras ya existentes.
- Una clase heredera es una versión más \itt{específica} que la heredada.
- Este es uno de los conceptos más importantes en OOP.


\centering\scalebox{0.6}{%
\begin{tikzpicture}
    \umlclass[x=0,y=0]{Persona}{}{}
    \umlclass[x=-2,y=-3]{Cliente}{}{}
    \umlclass[x=2,y=-3]{Aval}{}{}
    \umlinherit{Cliente}{Persona}
    \umlinherit{Aval}{Persona}
    % \umlrelation[style={tikzuml inherit style}]{Cliente}{Persona}
\end{tikzpicture}
}

### Relaciones entre Clases

\subsubtitleB{Ancestros y Descendientes}\newline

Para una clase en particular, se denomina:

\begin{description}[leftmargin=3em]
    \item[Ancestros:] Al conjunto de sus padres más los ancestros de éstos.
    \item[Descendientes:] Al conjunto de sus hijos más los descendientes de éstos.
\end{description}


### Relaciones entre Clases

\subsubtitleB{Ancestros y Descendientes}\newline

En la taxonomía anterior:

\hspace{4em}\taxonomyOrg[0.5]

- Tenemos que los ancestros de los mamíferos son: {vertebrados, animales, organismos}.
- Tenemos que los descendientes de las plantas son: {árboles, angiospermas, gimnospermas, helechos}.

### Relaciones entre Clases

\subsubtitleB{Herencia simple y múltiple}\newline

Si bien en la taxonomía anterior, toda clase tenía un solo padre, en algunos modelos nos podemos
encontrar con que una clase sea la especificación de dos o más clases unidas.\newline

\begin{columns}[t,onlytextwidth]
\column{0.6\textwidth}
{
\centering\scalebox{0.7}{%
\begin{tikzpicture}
    \umlclass[x=0,y=0]{Empleado}{}{}
    \umlclass[x=-3,y=-3]{Gerente}{}{}
    \umlclass[x=0,y=-3]{Vendedor}{}{}
    \umlclass[x=3,y=-3]{Ingeniero}{}{}
    \umlinherit{Gerente}{Empleado}
    \umlinherit{Vendedor}{Empleado}
    \umlinherit{Ingeniero}{Empleado}
    % \umlrelation[style={tikzuml inherit style}]{Cliente}{Persona}
\end{tikzpicture}
}
}
\column{0.4\textwidth}

{
    \vspace{-33mm}
\centering\scalebox{0.7}{%
\begin{tikzpicture}
    \umlclass[x=-2,y=0]{Gerente}{}{}
    \umlclass[x=2,y=0]{Vendedor}{}{}
    \umlclass[x=0,y=-3]{Gte Ventas}{}{}
    \umlinherit{Gte Ventas}{Gerente}
    \umlinherit{Gte Ventas}{Vendedor}
    % \umlrelation[style={tikzuml inherit style}]{Cliente}{Persona}
\end{tikzpicture}
}
}
\end{columns}


### Ejercicio 1

Elaborar un modelo básico de los clientes de un banco.

1. Un banco ofrece cuentas corrientes y cuentas vista a sus clientes.
1. Los clientes se relacionan con el banco a través de sus ejecutivos.
1. Debemos modelar todos los objetos involucrados en esta situación:
    - Propiedades de cada objeto.
    - Métodos de cada objeto.
1. No más de 8 clases.
1. No menos de 4 atributos por clase.
\vfill


\begin{center}\begin{customRoundedBox}{}%
\begin{itemize}
  \item Tienen 20 minutos.
  \item Ilustren el modelo con un diagrama.
\end{itemize}
\end{customRoundedBox}
\end{center}


### Ejercicio 2

- Considere una aplicación de su teléfono.
- Escriba la especificación de \itt{software} de esta aplicación.


\begin{center}\begin{customRoundedBox}{}%
\begin{itemize}
  \item Tienen 15 minutos.
  \item Escriba la especificación tal como usted sabe hacerlo ahora.
\end{itemize}
\end{customRoundedBox}
\end{center}

### Relaciones entre Clases

\subsubtitleB{Agregación}\newline

\columnsbegin

\column{.6\textwidth}

En la \textcolor{blue}{agregación} tenemos que un objeto forma parte
de un ``equipo'' de colaboradores.

- Los colaboradores tienen vida propia.
- La creación o destrucción del objeto destino no implica que lo mismo
suceda con los colaboradores.

\column{.4\textwidth}

\centering\scalebox{0.6}{%
\begin{tikzpicture}
    \umlclass[x=0,y=0]{Teatro}{}{}
    \umlclass[x=0,y=-3]{Película}{}{}
    \umlaggreg[mult1=*,mult2=*]{Teatro}{Película}
\end{tikzpicture}
}

\columnsend

### Relaciones entre Clases

\subsubtitleB{Composición}\newline

\columnsbegin

...

\column{.5\textwidth}

\centering\scalebox{0.6}{%
\begin{tikzpicture}
    \umlclass[x=0,y=0]{Teatro}{}{}
    \umlclass[x=0,y=-3]{Boletería}{}{}
    \umlcompo[mult1=1,mult2=1]{Teatro}{Boletería}
\end{tikzpicture}
}

\columnsend

### Relaciones entre Clases

\subsubtitleB{Agregación y Composición}\newline


\usebox{\umlAgregComp}\par

Agregación:

- Un polígono y un círculo pueden tener un estilo definido.
- La \textcolor{blue}{existencia} del estilo es independiente de
qué objetos estén asociados a él.

### Relaciones entre Clases

\subsubtitleB{Agregación y Composición}\newline

\usebox{\umlAgregComp}\par

Composición:

- Una figura (polígono o círculo) está compuesta por puntos como parte
de su descripción.
- Al crear una figura, se crean los puntos necesarios.
- Al eliminar una figura, se eliminan en cascada sus puntos.

### Relaciones entre Clases

\subsubtitleB{Agregación y Composición}\newline

¿Cómo modelaríamos un automóvil, considerando sus componentes principales?

- ¿Cuáles son los componentes principales de un automóvil?
- Aquellos objetos, ¿existen necesariamente desde antes de que el vehículo
esté armado?
- ¿Qué tan fuerte es el vínculo entre estos componentes y el vehículo?
- Al momento de ser destruido un vehículo, ¿necesariamente se destruyen
sus componentes?

### Relaciones entre Clases

\subsubtitleB{Navegabilidad}\newline

...


### UML

\subsubtitleB{Perspectivas en DC.}\newline

- Conceptual
    - Indican las responsabilidades de cada clase, más que su interfaz.
- Especificación
- Implementación
