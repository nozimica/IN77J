# Diagrama de Clases en profundidad

## Repaso: las clases en un D.C.

### Repaso: las clases en un D.C.

\subsubtitleB{Una clase en un diagrama muestra:}

- El nombre de la clase.
- El listado de sus atributos.
    - Indicando su visibilidad.
    - Indicando su tipo de datos.
- El listado de sus operaciones.
    - Indicando su visibilidad.
    - Indicando el tipo de dato de sus parámetros.
        - Opcionalmente el nombre del parámetro.
    - Indicando el tipo de dato que retorna.

\centering\scalebox{0.5}{%
\begin{tikzpicture}
    \umlclass[x=0,y=0]{Vuelo}%
      {--númDeVuelo : Integer\\ --horaSalida : Date\\ --duraciónVuelo : Integer}%
      {+atrasarVuelo(Integer) : Date\\ +obtHoraLlegada() : Date}
    \umlclass[x=8,y=0]{CuentaBancaria}%
      {--dueño : Cliente\\ --balance : Money\\ --apertura : Date}%
      {+depositar(monto : Money) : Boolean\\ +sacar(monto : Money) : Boolean}
\end{tikzpicture}
}


## Repaso: las relaciones en un D.C.

### Repaso: las relaciones en un D.C.

\subsubtitleB{Cardinalidad o multiplicidad}

- Es conveniente detenernos en la cardinalidad.
    - Se usa en varios tipos de relaciones entre clases.

- Valores usados:

\begin{center}
\begin{footnotesize}
\begin{tabular}{cl}
\toprule
\bld{Notación} & \bld{Lectura} \\
\midrule
\bld{1} & Exactamente uno. \\
\bld{n} & Exactamente n. \\
\bld{0..1} & Entre cero y uno. \\
\bld{n..m} & Entre n y m (ambos inclusive). \\
\bld{*} & Cero, uno o cualquier otro número. \\
\bottomrule
\end{tabular} 
\end{footnotesize}
\end{center}


### Repaso: las relaciones en un D.C.

\subsubtitleB{Cardinalidad o multiplicidad}

- Algunas interpretaciones útiles:

\begin{center}
\begin{footnotesize}
\begin{tabular}{cl}
\toprule
\bld{Notación} & \bld{Interpretación} \\
\midrule
Límite inferior \bld{cero} & Es opcional. \\
Límite inferior $\geq$ \bld{uno} & Es obligatorio. \\
\bld{1} & Es ``univaluado''. \\
\bld{*} & Es ``multivaluado'' (implica una lista). \\
\bottomrule
\end{tabular} 
\end{footnotesize}
\end{center}


### Repaso: las relaciones en un D.C.

\subsubtitleB{Asociación}

- Una clase tiene como \bld{propiedad a una instancia de otra clase}.
    - Puede ser una instancia de la misma clase también.
- Se usa una línea continua.
    - Si tiene flecha, indica la \bld{dirección} de la asociación.
- Puede indicarse la \bld{cardinalidad} de la asociación.
- Suele indicarse el nombre que la propiedad tiene en la clase ``contenedora''.

\tikzinlinec[0.5]{%
    \umlclass[x=0,y=0]{Orden}
        {--fecha : Integer\\--prepagada : Boolean}
        {+enviaEmail() : Boolean}
    \umlclass[x=10,y=0]{Cliente}
        {--nombre : String\\--dirección : String\\--antigüedad : Integer}
        {}
    \umluniassoc[mult1=*,pos1=0.1,mult2=1,arg2=Emisor,pos2=0.85]{Orden}{Cliente}
}

### Repaso: las relaciones en un D.C.

\subsubtitleB{Dirección en una asociación}

Fundamentalmente tenemos dos casos:

- Ambas clases son ``conscientes'' de la asociación: \bld{bi-direccional}.

\tikzinlinec[0.5]{%
    \umlclass[x=0,y=0]{Persona}
        {--nombre : String\\--edad : Integer\\--sexo : Char}
        {}
    \umlclass[x=10,y=0]{Automóvil}
        {--color : String\\--marca : String\\--año : Integer}
        {}
    \umlbiassoc[mult1=0..1,arg1=dueño,pos1=0.15,mult2=*,pos2=0.85]{Persona}{Automóvil}
}

- Implica algún mecanismo de ``sincronización''.
- Se diagrama \bld{con dos flechas}, o también \bld{con ninguna}.

### Repaso: las relaciones en un D.C.

\subsubtitleB{Dirección en una asociación}

En este caso de \bld{asociación bi-direccional}:

\begin{center}
\scalebox{0.5}{%
\begin{tikzpicture}
    \umlclass[x=0,y=0]{Persona}
        {--nombre : String\\--edad : Integer\\--sexo : Char}
        {}
    \umlclass[x=10,y=0]{Automóvil}
        {--color : String\\--marca : String\\--año : Integer}
        {}
    \umlbiassoc[mult1=0..1,arg1=dueño,pos1=0.15,mult2=*,pos2=0.85]{Persona}{Automóvil}
\end{tikzpicture}
}
\end{center}

- Una persona puede tener ninguno, uno o muchos automóviles.
    - Entre sus atributos habrá una lista de automóviles.
- Un automóvil tiene un solo dueño, o puede no tener ninguno aún.
    - Tendrá un atributo llamado \bld{dueño}, pues así está indicado en el diagrama.

### Repaso: las relaciones en un D.C.

\subsubtitleB{Dirección en una asociación}

Fundamentalmente tenemos dos casos:

- Sólo una clase es ``consciente'' de esta relación: \bld{uni-direccional}.

\begin{center}
\scalebox{0.5}{%
\begin{tikzpicture}
    \umlclass[x=0,y=0]{ReporteCuentasSobregiradas}
        {--fechaGeneración : Date}
        {+actualizar() : Boolean}
    \umlclass[x=10,y=0]{CuentaBancaria}%
      {--balance : Money\\ --apertura : Date}%
      {+depositar(monto : Money) : Boolean\\ +sacar(monto : Money) : Boolean}
    \umlclass[x=10,y=-5]{Cliente}
        {--nombre : String\\--dirección : String\\--antigüedad : Integer}
        {}
    \umluniassoc[mult2=*,arg2=ctasSobreg,pos2=0.7]{ReporteCuentasSobregiradas}{CuentaBancaria}
    \umlbiassoc[mult1=0..1,arg1=dueño,pos1=0.2,mult2=*,pos2=0.8]{Cliente}{CuentaBancaria}
\end{tikzpicture}
}
\end{center}

### Repaso: las relaciones en un D.C.

\subsubtitleB{Dirección en una asociación}

En este caso de \bld{asociación uni-direccional}:

\begin{center}
\scalebox{0.5}{%
\begin{tikzpicture}
    \umlclass[x=0,y=0]{ReporteCuentasSobregiradas}
        {--fechaGeneración : Date}
        {+actualizar() : Boolean}
    \umlclass[x=10,y=0]{CuentaBancaria}%
      {--dueño: Cliente\\ --balance : Money\\ --apertura : Date}%
      {+depositar(monto : Money) : Boolean\\ +sacar(monto : Money) : Boolean}
    \umluniassoc[mult2=*,arg2=ctasSobreg,pos2=0.7]{ReporteCuentasSobregiradas}{CuentaBancaria}
\end{tikzpicture}
}
\end{center}

- Los reportes tienen un listado de las cuentas involucradas.
- Las cuentas no tienen forma de entregar el o los reportes donde son mencionadas.
    - Esto es claramente una decisión de diseño.
    - Permite aliviar el modelo de restricciones de sincronización.
        - Se hace en los casos en que no es tan relevante que una clase sepa que ``están
        hablando de ella en otra parte''.

### Repaso: las relaciones en un D.C.

\subsubtitleB{Asociaciones reflexivas}

- Una instancia de una clase puede perfectamente tener como atributo una instancia distinta de ella misma.

\begin{center}
\scalebox{0.6}{%
\begin{tikzpicture}
    \umlclass[x=0,y=0]{Persona}
        {--nombre : String\\--edad : Integer\\--sexo : Char}
        {}
    \umluniassoc[mult1=1,arg1=padre,pos1=0.1,angle1=-90,angle2=-180,loopsize=40mm,mult2=*,pos2=0.9,arg2=hijo]{Persona}{Persona}
\end{tikzpicture}
}
\end{center}

### Repaso: las relaciones en un D.C.

\subsubtitleB{Clases de asociación}

- Hay veces que una asociación puede tener atributos en sí misma.
- En algunos casos, conviene encapsular esos atributos en una nueva clase.


\begin{center}
\scalebox{0.5}{%
\begin{tikzpicture}
    \umlclass[x=0,y=0,width=10em]{Cliente}
        {--nombre : String\\--dirección : String\\--antigüedad : Integer}
        {}
    \umlclass[x=10,y=0,width=10em]{Automóvil}
        {--color : String\\--marca : String\\--año : Integer}
        {}
    \umlbiassoc[name=assoc,mult1=0..1,pos1=0.15,mult2=*,pos2=0.85]{Cliente}{Automóvil}
    \umlassocclass[x=5,y=-4,width=10em]{Arriendo}{assoc-1}{--fechaIni : Date\\ --fechaFin : Date\\ --precio : Money}{}
\end{tikzpicture}
}
\end{center}

- Ojo con la línea segmentada.

### Repaso: las relaciones en un D.C.

\subsubtitleB{Herencia}

\begin{description}
    \item[Generalización:] Se tiene cuando dadas dos o más clases con características y/o comportamiento en común:
    \begin{itemize}
        \item Modelamos una nueva clase que refleje aquello que ellas comparten.
    \end{itemize}
\vfill
    \item[Especialización:] Cuando una clase está comprendiendo muchas características, no siempre presentes en sus instancias:
    \begin{itemize}
        \item Modelamos varias clases, donde cada una se encarga de esas características específicas.
        \item Las clases producidas deben tener sentido en nuestro modelo.
    \end{itemize}
\end{description}

### Repaso: las relaciones en un D.C.

\subsubtitleB{Herencia}

Por ejemplo:

- Nuestra aplicación bancaria maneja muchos datos de clientes y de avales.
    - Vemos que sus características son prácticamente las mismas.
    - Muchos de sus comportamientos son iguales.
    - Generamos una generalización llamada... \bld{Persona}.

\tikzinlinec{%
    \umlclass[x=0,y=0]{Persona}{}{}
    \umlclass[x=-2,y=-3]{Cliente}{}{}
    \umlclass[x=2,y=-3]{Aval}{}{}
    \umlinherit{Cliente}{Persona}
    \umlinherit{Aval}{Persona}
}

### Repaso: las relaciones en un D.C.

\subsubtitleB{Herencia}

Por ejemplo:

- Nuestra aplicación maneja información de los empleados en una empresa (quizás es un ERP).
    - Comenzamos pensando en los empleados en general.
    - Pronto nos damos cuenta que no todas las características ni comportamientos aplican
    a todos por igual.
    - Hay cosas que son específicas: creamos las clases \bld{Gerente} y \bld{Vendedor}.

\tikzinlinec[0.6]{%
    \umlclass[x=0,y=0]{Empleado}{}{}
    \umlclass[x=-2,y=-3]{Gerente}{}{}
    \umlclass[x=2,y=-3]{Vendedor}{}{}
    \umlinherit{Gerente}{Empleado}
    \umlinherit{Vendedor}{Empleado}
}

### Repaso: las relaciones en un D.C.

\subsubtitleB{Dependencia}

- Útil para dejar explícito que una clase A \texthigh{depende} de otra B.
    - Si B cambia, A \texthigh{posiblemente se verá afectada} y deberá ser revisada.

- Las asociaciones (y todos sus tipos), herencia, interfaces y clases abstractas implican
un grado de dependencia.

- Cuando esta dependencia no es expresable de una manera más específica (como aquellas mencionadas arriba),
se utiliza una flecha con una línea segmentada.

\tikzinlinec[0.5]{%
    \umlclass[x=0]{PlanificaciónCursos}{}{+agrega(Curso) : boolean\\+elimina(Curso) : boolean}
    \umlclass[x=6]{Curso}{}{}
    \umldep{PlanificaciónCursos}{Curso}
}

## Repaso: las interfaces en un D.C.

### Repaso: las interfaces en un D.C.

\subsubtitleB{Interfaces}

- En OOP, el concepto de \texthigh{Interfaz} es muy importante.

- Al hablar de él, nos referimos a un \texthigh{grupo de operaciones}.
    - A este grupo la damos un nombre.

- Una clase se dice que \texthigh{implementa} una o varias interfaces.

- Si una clase implementa una interfaz, entonces:
    - Se \itt{compromete} a darle una implementación (métodos) a cada operación de
    la interfaz.

\begin{rboxx}{}
Una interfaz le da un caracter específico al comportamiento de las instancias de la clase.
\end{rboxx}

### Repaso: las interfaces en un D.C.

\subsubtitleB{Interfaces: ejemplo}

- Imaginémonos que en nuestro modelo están las clases \bld{Persona}, \bld{Avión} y \bld{Reserva}.

- Por requisitos del cliente, es necesario poder comparar distintas personas entre sí, dependiendo de su edad.
    - Y nos piden comparar aviones según su cantidad máxima de asientos.
    - Y también nos piden comparar reservas según la antelación con que fueron hechas.

- Vale decir, estas tres clases deben ser capaces de proveer la operación \texttt{compararCon(Objeto)}.

### Repaso: las interfaces en un D.C.

\subsubtitleB{Interfaces: ejemplo}

- Para establecer el caracter de ``Comparable'' que tienen los objetos de estas tres clases:
    - Definimos una \texthigh{interfaz} llamada \texthigh{Comparable}.
    - Esta interfaz declara una operación llamada \texttt{compararCon(Objeto)}.

\tikzinlinec{%
    \umlemptyclass[x=-3,y=1]{Objeto}
    \umlclass[type=interface,x=3,y=1]{Comparable}{}{+compararCon(Objeto)}
    \umlemptyclass[x=-4,y=-3]{Persona}
    \umlemptyclass[x=0,y=-3]{Avión}
    \umlemptyclass[x=4,y=-3]{Reserva}
    \umlinherit{Persona}{Objeto}
    \umlinherit{Avión}{Objeto}
    \umlinherit{Reserva}{Objeto}
    \umlimpl{Persona}{Comparable}
    \umlimpl{Avión}{Comparable}
    \umlimpl{Reserva}{Comparable}
}

### Repaso: las interfaces en un D.C.

\subsubtitleB{Interfaces}

\vspace{-1.5em}
\columnsbegin
\column{0.6\textwidth}

- Ojo: una interfaz puede heredar de otra interfaz.
- Notar los tipos de flecha que se usan:
    - \texthigh{Segmentado}: entre una interfaz y una clase
    - \texthigh{Continuo}: entre una interfaz y otra interfaz.
- Las interfaces definen el comportamiento de otras clases:
    - Ellas no pueden crear objetos propios.


\column{0.4\textwidth}

\tikzinlinec{%
    \umlclass[type=interface,y=0]{Unible}{}{+unirCon(Objeto)}
    \umlclass[type=interface,y=-4]{Comparable}{}{+compararCon(Objeto)}
    \umlemptyclass[y=-8]{Conjunto}
    \umlinherit{Comparable}{Unible}
    \umlimpl{Conjunto}{Comparable}
}

\columnsend

### Repaso: las interfaces en un D.C.

\subsubtitleB{Interfaces: ejemplo}

- Tenemos un equipo de música que ofrece dos maneras de \texthigh{interactuar} con
el usuario:
    - Mediante su panel frontal.
    - A distancia mediante un control remoto.

- Pero independiente de qué se use, necesitamos que nos ofrezcan las mismas acciones.
    - Necesitamos una \texthigh{interfaz} común.

\columnsbegin
\column{0.3\textwidth}
\centering\includegraphics[width=20mm]{icons/554-emoji_android_mobile_phone.png}
\column{0.3\textwidth}
\centering\includegraphics[width=20mm]{icons/562-emoji_android_radio.png}
\column{0.1\textwidth}
\column{0.3\textwidth}

\vspace{-5em}
1. Play/Pause
1. Stop
1. Next
1. Previous

\columnsend

## Clases abstractas

### Clases abstractas

\subsubtitleB{¿Por qué aparece el concepto de clase abstracta?}

- Ya sabemos que la \bld{herencia} nos permite \texthigh{agrupar comportamientos y/o características} que son
compartidos por varias clases.
    - Esto define una superclase y una o varias subclases.
- Pero \texthigh{en algunos casos}, si bien sabemos el \texthigh{comportamiento que deberán tener las subclases}:
    - También tenemos claro que ese \texthigh{comportamiento no será idéntico entre varias subclases} ``hermanas''.

### Clases abstractas

\subsubtitleB{Clase abstracta: ejemplo}

Imaginémonos que estamos modelando figuras geométricas: \bld{Círculo}, \bld{Cilindro} y \bld{Rectángulo}.

- De inmediato podemos pensar en una superclase llamada ---por ejemplo--- \bld{FiguraGeométrica}.
    - Dentro de ella encapsularemos características y comportamientos comunes:

\tikzinlinec{%
    \umlclass[x=0,y=0]{FiguraGeométrica}%
      {--colorLínea : String\\ --colorRelleno: String}%
      {// setters, getters\\+calculaÁrea() : double\\+calculaPerímetro() : double}%
}

### Clases abstractas

\subsubtitleB{Clase abstracta: ejemplo}

\vspace{-1.5em}
\columnsbegin
\column{0.5\textwidth}

- Tenemos un problema:
    - Sabemos que tanto a Círculos como a Rectángulos se les puede calcular el área y el perímetro.
    - Pero ambos cálculos son distintos dependiendo de la figura.

\column{0.5\textwidth}
\tikzinlinec[0.55]{%
    \umlclass[x=0,y=5]{FiguraGeométrica}%
      {--colorLínea : String\\ --colorRelleno: String}%
      {// setters, getters\\+calculaÁrea() : double\\+calculaPerímetro() : double}%
    \umlclass[x=-2,y=0]{Círculo}%
      {--radio : double}%
      {// setters, getters}%
    \umlclass[x=2,y=0]{Rectángulo}%
      {--ancho : double\\--alto : double}%
      {// setters, getters}%
    \umlclass[x=0,y=9,simple]{Object}{}{}%
    \umlinherit{FiguraGeométrica}{Object}
    \umlinherit{Círculo}{FiguraGeométrica}
    \umlinherit{Rectángulo}{FiguraGeométrica}
}

\columnsend

### Clases abstractas

\subsubtitleB{Clases abstractas en el D.C.}

- Una operación que está \texthigh{declarada pero no implementada} se define como \bld{abstracta}.

- Al suceder esto, toda la clase que la contiene también es abstracta.
    - En el diagrama de clases se escribe en cursivas:
        - El nombre de la clase.
        - El nombre de las operaciones abstractas.

\vspace{-1.5em}
\columnsbegin
\column{0.6\textwidth}

- Al igual que las interfaces, una clase abstracta \texthigh{no puede crear objetos}.

\column{0.4\textwidth}
\tikzinlinec[0.5]{%
    \umlclass[x=0,y=5,type=abstract]{FiguraGeométrica}%
      {--colorLínea : String\\ --colorRelleno: String}%
      {// setters, getters\\\itt{+calculaÁrea() : double}\\\itt{+calculaPerímetro() : double}}%
}
\columnsend

### Clases abstractas

\subsubtitleB{Ejemplos}

\vspace{-1.5em}
\columnsbegin

\column{0.5\textwidth}

- Para el ejemplo anterior, debemos:
    - Escribir en cursiva el nombre de la clase abstracta.
    - Escribir en cursiva sus operaciones abstractas.
    - En cada clase que extienda a la clase abstracta, escribir
    esas operaciones como ya implementadas.

\column{0.5\textwidth}

\tikzinlinec[0.5]{%
    \umlclass[x=0,y=5,type=abstract]{FiguraGeométrica}%
      {--colorLínea : String\\ --colorRelleno: String}%
      {// setters, getters\\\itt{+calculaÁrea() : double}\\\itt{+calculaPerímetro() : double}}%
    \umlclass[x=-3,y=0]{Círculo}%
      {--radio : double}%
      {// setters, getters\\+calculaÁrea() : double\\+calculaPerímetro() : double}%
    \umlclass[x=3,y=0]{Rectángulo}%
      {--ancho : double\\--alto : double}%
      {// setters, getters\\+calculaÁrea() : double\\+calculaPerímetro() : double}%
    \umlclass[x=0,y=9,simple]{Object}{}{}%
    \umlinherit{FiguraGeométrica}{Object}
    \umlinherit{Círculo}{FiguraGeométrica}
    \umlinherit{Rectángulo}{FiguraGeométrica}
}

\columnsend

### Clases abstractas

\subsubtitleB{Ejemplos}

\tikzinlinec[0.55]{%
    \umlclass[x=0,type=interface]{Collection}%
      {}%
      {+equals()\\+add()}%
    \umlclass[x=4,type=interface]{List}%
      {}%
      {+get()}%
    \umlclass[x=8,type=abstract]{AbstractList}%
      {}%
      {+equals()\\+add()\\\itt{+get()}}%
    \umlclass[x=12]{ArrayList}%
      {}%
      {+get()\\+add()}%
    \umlinherit{List}{Collection}
    \umlimpl{AbstractList}{List}
    \umlinherit{ArrayList}{AbstractList}
}

- Tenemos un \bld{ArrayList} que es una lista de elementos.
- Esta lista implementa dos interfaces:
    - \bld{Collection}: una colección de elementos.
    - \bld{List}: que permite obtener un elemento.
- Tenemos una clase abstracta \bld{AbstractList}, la que solamente
implementa las operaciones \bld{equals()} y \bld{add()}, dejando \bld{get()}
como abstracta.
- Finalmente, \bld{ArrayList} implementa \bld{get()}, y reimplementa la operación
\bld{add()}.

