# Arquitectura

## Definición de arquitectura

### Definición de arquitectura

\subsubtitleB{¿A qué nos referimos con arquitectura?}
\vspace{-0.5em}
\buildrboxx{}
Establece la estructura global de una aplicación.
\finishrboxx

- Define el marco para el diseño y el desarrollo de un Sistema de Información.
    - Es un modelo que contiene:
        - Un conjunto de componentes.
        - Sus relaciones.
    - Provee una visión global para simplificar la comprensión de la solución modelada.

### Definición de arquitectura

\subsubtitleB{Características de las arquitecturas}

- El nivel de abstracción es más alto que el del diseño.
- Se puede refinar en sucesivas iteraciones.
    - Va mostrando cada vez \texthigh{más detalles}.
    - Así se van modelando cada uno de los módulos que componen la solución.

### Definición de arquitectura

\subsubtitleB{Características de las arquitecturas}

- Primero la arquitectura, después el diseño:
    - El nivel de abstracción de la arquitectura permite manejar de mejor manera la complejidad
    del problema.
    - Una vez que esta complejidad esté bajo control, se profundiza en el
    problema hasta llegar a la solución.

- Se enfoca en diversos aspectos del problema.
    - La estructura interna.
    - La estructura de los datos.
    - La estructura informática.

### Definición de arquitectura

\subsubtitleB{Arquitectura lógica: enfoque \itt{Top-Down}}

- Es una de las maneras más típicas de abordar un problema.
- Al principio se tiene una visión global del sistema.
    - Se lo ve como un todo unitario.
- Luego, se identifica sus componentes internos más primarios.
- Más adelante, se sigue \texthigh{profundizando} en el análisis,
caracterizando más detalladamente cada uno de esos componentes.
- Se profundiza tantas veces como sea necesario:
    - Para obtener una granularidad suficiente como para describir los
    detalles escenciales de cada componente.

### Definición de arquitectura

\subsubtitleB{Arquitectura lógica: enfoque \itt{Top-Down}}

\def\distFlow{2em}
\begin{center}\begin{tikzflowchart}
  \tikzstyle{flowElemSp} = [normalfig, rounded corners=3mm, text width=11mm, line width=1.5pt, minimum height=1.5em]
  \tikzstyle{flowSp2} = [flow2, flowElemSp, draw=black!50!blue]
  \tikzstyle{flowSp3} = [flow3, flowElemSp, draw=black!50!green]
  \tikzstyle{flowSp2in} = [flow2, flowElemSp, draw=black!50!blue]
  \tikzstyle{flowSp3in} = [flow3, flowElemSp, draw=black!50!green]
  \node (a1) [flowSp3,text width=40mm, text height=2em] {};
    \node[anchor=center] at (a1.center) {Sistema};
  \node (a2) [flowSp3, text width=80mm, text height=3em, below=\distFlow of a1] {};
    \node[anchor=north] at (a2.north) {Sistema};
    \node [flowSp2, anchor=west] at ($ (a2.west)+(1em,-0.5em) $) {Comp1};
    \node [flowSp2, anchor=center] at ($ (a2.center)+(0,-0.5em) $) {Comp2};
    \node [flowSp2, anchor=east] at ($ (a2.east)+(-1em,-0.5em) $) {Comp3};
  \node (a2a) [flowSp3, text width=34mm, text height=3em, below=\distFlow of a2] {};
    \node[anchor=north] at (a2a.north) {Sistema, Comp2};
    \node [flowSp2, anchor=west] at ($ (a2a.west)+(0.5em,-0.5em) $) {Comp2a};
    \node [flowSp2, anchor=east] at ($ (a2a.east)+(-0.5em,-0.5em) $) {Comp2b};
  \node (a2b) [flowSp3, text width=34mm, text height=3em, left=0.6em of a2a] {};
    \node[anchor=north] at (a2b.north) {Sistema, Comp1};
    \node [flowSp2, anchor=west] at ($ (a2b.west)+(0.5em,-0.5em) $) {Comp1a};
    \node [flowSp2, anchor=east] at ($ (a2b.east)+(-0.5em,-0.5em) $) {Comp1b};
  \node (a2c) [flowSp3, text width=34mm, text height=3em, right=0.6em of a2a] {};
    \node[anchor=north] at (a2c.north) {Sistema, Comp3};
    \node [flowSp2, anchor=west] at ($ (a2c.west)+(0.5em,-0.5em) $) {Comp3a};
    \node [flowSp2, anchor=east] at ($ (a2c.east)+(-0.5em,-0.5em) $) {Comp3b};
  \draw[arrow,line width=2pt] (a1) -- (a2);
  \draw[arrow,line width=2pt] (a2) -- (a2a);
  \draw[arrow,line width=2pt] (a2) -- (a2b);
  \draw[arrow,line width=2pt] (a2) -- (a2c);
  \coordinate[right=30mm of a1] (c1) {};
  \coordinate[below=35mm of c1] (c2) {};
  \draw[arrow,draw=red!30,line width=3pt] (c1) -- (c2) node [midway,above,sloped] {\textcolor{red}{\textbf{Refinamiento}}};
\end{tikzflowchart}\end{center}

### Definición de arquitectura

\subsubtitleB{Arquitectura lógica: enfoque \itt{Top-Down}}

- ¿Cuál será el término del refinamiento?
    - Una descripción muy detalada de la solución.
    - Una descripción que caracterice las componentes más internas de cada módulo.
        - Corresponde al \texthigh{diagrama de clases}

\vfill

- Vale decir:
    - \bld{En un extremo} tenemos la \texthigh{arquitectura}.
    - \bld{En el otro extremo} tenemos el \texthigh{diseño de clases}.

\buildrboxx{}
Ojo: El límite entre ambos es difuso.
\finishrboxx

### Definición de arquitectura

\subsubtitleB{Arquitectura lógica: enfoque \itt{Top-Down}}

- Se hace útil el tener alguna manera de guiar el proceso de refinamiento.

\buildrboxx{}
Sí existen, y se denominan patrones de arquitectura.
\finishrboxx

- Son genéricos.
- La elección de uno u otro depende del problema que se está modelando.
- Su enfoque es \texthigh{asignar responsabilidades específicas} a los componentes
resultantes del refinamiento aplicado.


### Arquitecturas en acción

\subsubtitleB{Aspectos de una aplicación}

- Toda aplicación es posible de describir como la interacción de diversos aspectos.
    - No siempre son los mismos.
    - Una buena definición de aspectos redunda en un diseño más flexible.

- Bondades de dividir un problema, en este caso en aspectos:
    - Ya hemos visto que todo problema subdividido es más fácil de abordar.
    - Cada aspecto puede desarrollarse y evolucionar de manera independiente.
    - Incluso es posible reutilizar aspectos ya modelados y entendidos en otras
    aplicaciones similares.

### Arquitecturas en acción

\subsubtitleB{Aspectos de una aplicación}

\begin{center}\begin{tikzflowchart}
  \draw[arrow,draw=red!30,line width=3pt,line cap=round] (0,0) -- (7,0) node [very near start,below] {\textcolor{red}{\textbf{menor}}} node [very near end,below] {\textcolor{red}{\textbf{mayor}}} node [midway,below=4mm] {\textcolor{red}{\textbf{separación de aspectos}}};
  \draw[arrow,draw=red!30,line width=3pt,line cap=round] (0,0) -- (0,5) node [very near start,above,sloped] {\textcolor{red}{\textbf{menor}}} node [very near end,above,sloped] {\textcolor{red}{\textbf{mayor}}} node [midway,above=4mm,sloped] {\textcolor{red}{\textbf{simplificación}}};
  \draw[arrow,draw=blue!30,line width=3pt] (1,4) -- (6,1) node [very near start,above,sloped] {\textcolor{blue}{\textbf{arquitectura}}};
  \draw[arrow,draw=green!60!black,line width=3pt] (1,1) -- (6,4) node [very near end,above,sloped] {\textcolor{green}{\textbf{diseño}}};
\end{tikzflowchart}\end{center}


### Arquitecturas en acción

\subsubtitleB{Aspectos de una aplicación}

- Una separación muy utilizada divide toda la aplicación en tres grandes aspectos:

\begin{center}\begin{tikzflowchart}
  \node [startstop, text width=20mm, minimum height=2em] (mvc1) {Presentación};
  \node [startstop, text width=20mm, minimum height=2em] at (2,-1.5) (mvc2) {Lógica};
  \node [startstop, text width=20mm, minimum height=2em] at (4,0) (mvc3) {Persistencia};
  \draw [arrow,<->,line width=2pt] (mvc1) -- (mvc2);
  \draw [arrow,<->,line width=2pt] (mvc3) -- (mvc2);
\end{tikzflowchart}\end{center}

\pause

\begin{description}
    \item[Presentación:] Permite la interacción entre el sistema y el exterior.
    \item[Lógica:] Procesa las acciones y los datos desde y hacia el exterior.
    \item[Persistencia:] Conserva los datos para poder usarlos más tarde.
\end{description}

### Arquitecturas en acción

\subsubtitleB{Aspectos de una aplicación}

- ¿Cómo reflejamos lo anterior al momento de diseñar las clases?
    - ¿Conviven todos los aspectos dentro de una misma clase?

\pause

- Lo más sano es que cada aspecto sea interpretado por una o varias clases.
- Sin mezclar dos o más aspectos dentro de una de ellas.

### Arquitecturas en acción

\subsubtitleB{Arquitectura en capas}

- En este caso el particionado de la aplicación se basa en \texthigh{capas concéntricas}.
    - Cada capa es un \bld{nivel}.
    - Elementos en un mismo nivel tienen responsabilidades similares.
    - Existe una jerarquía:
        - Los elementos de un nivel atienden las peticiones provenientes del nivel superior.

### Arquitecturas en acción

\subsubtitleB{Arquitectura en capas}

\begin{center}\begin{tikzflowchart}
\coordinate (O) at (0,0);
\draw[fill=gray!30] (O) circle (3);
\draw[fill=gray!50] (O) circle (2);
\draw[fill=gray!70] (O) circle (1);
\node at (0,0) {capa 1};
\node at (1.5,0) {capa 2};
\node at (2.5,0) {capa 3};
\umlactor[x=-5,y=0,scale=0.7]{usuario 1};
\umlactor[x=5,y=1,scale=0.7]{usuario 2};
\umlactor[x=5,y=-1,scale=0.7]{usuario 3};
\draw [arrow,line width=3pt] (usuario 1) -- (-3.2,0);
\draw [arrow,line width=3pt] (usuario 2) -- (3.2,0.7);
\draw [arrow,line width=3pt] (usuario 3) -- (3.2,-0.7);
\end{tikzflowchart}\end{center}

### Arquitecturas en acción

\subsubtitleB{Arquitectura en capas}

- Los aspectos que vimos anteriormente sí son compatibles con la arquitectura en capas.
- De hecho, podemos definir una arquitectura en 3 capas:
    - Presentación.
    - Lógica.
    - Persistencia.

\vfill
\pause

- \bld{Importante:} Los usuario solamente \texthigh{interactúan con la capa de presentación}.
- Podemos pensar que la capa de \texthigh{persistencia} no requiere la atención de ninguna otra.

### Arquitecturas en acción

\subsubtitleB{Arquitectura en capas: ¿qué hay en cada una de ellas?}

- \texthigh{Presentación:} clases que proveen la capacidad de interactuar con el usuario:
    - Capturando datos de entrada.
    - Presentando datos de salida.
- \texthigh{Lógica:} clases que:
    - Describen los objetos que contienen la lógica de negocio para
satisfacer cada uno de los \texthigh{casos de uso} ya definidos.
    - Clases que permiten comunicarse con los datos almacenados en la capa de persistencia.
- \texthigh{Persistencia:} clases que implementan el almacenamiento de los datos, en el o los formatos
que sean necesarios: base de datos, XML, JSON, texto, video, etc.

