## Descriptors

### Descriptors

\subsubtitleB{Definición de \itt{descriptors}}

Junto con los modelos de clase, aparece el concepto de \texthigh{\itt{descriptors}}:\newline

\begin{description}
\item[Full descriptor:] Es la descripción completa necesaria para describir a un objeto, considerando
todos sus atributos, operaciones, métodos y asociaciones.
\item[Segment descriptor:] Son los elementos que efectivamente se declaran en un modelo o en el código
(clases, por ejemplo) y que contienen las propiedades heredables que son:
    \begin{footnotesize}
    \begin{itemize}
        \item Atributos (\textsc{ATT}).
        \item Operaciones.
        \item Métodos.
        \item Participación en asociaciones (pseudoatributos).
    \end{itemize}
    \end{footnotesize}
\end{description}

### Full descriptors

\subsubtitleB{¿Para qué sirven?}

- Muestran un resumen de las características más importantes de las clases.
    - Características desde un punto de vista estructural, no conceptual.
- Son especialmente útiles cuando existen relaciones de herencia.
    - Permiten ver cómo los métodos \texthigh{sobreescritos} ocultan a los de
    las clases ascendientes.
