\documentclass[11pt]{article}

\newcommand{\hmwkCode}{Control 1}
\newcommand{\hmwkDueDate}{S\'{a}bado 3 de octubre de 2015}
\newcommand{\hmwkClassTime}{Tiempo total: 40 minutos. Con apuntes personales.}
\usepackage{in77j.exercise}

\makeatletter 
%\newcommand\mynobreakpar{\par\nobreak\@afterheading} 
\@beginparpenalty=10000
\makeatother

\setlength{\parskip}{0.7\baselineskip}
\begin{document}
\begin{spacing}{1.2}
\thispagestyle{empty}
\evaluationTitle

\newcommand{\separatorLine}{\begin{center}\rule{.6\textwidth}{1pt}\end{center}}

\begin{Problem}

    \bld{Considere el proceso de venta de entradas para conciertos.}

    Existe una empresa encargada de vender entradas para cada uno de los conciertos que
    tenga disponibles en un momento dado.

    Este servicio se ofrece en un solo local físico, el cual cuenta con tres cajas que ---operadas
    por sendas cajeras o cajeros--- atiende a los clientes.
    
    Estos clientes obtienen un número de atención correlativo al momento de llegar al local. En todo
    momento hay un funcionario de la empresa encargado de avisar a los clientes cuando se desocupe
    una caja. Para esto, evidentemente, se hace uso del número correlativo ya mencionado.

    Se ofrece como medio de pago tanto el efectivo como tarjetas de débito o crédito. Para este último
    medio de pago no se considera la compra en cuotas.

    Es importante considerar que las entradas a todo concierto están limitadas en su cantidad.

\begin{ProbPart}[difficulty={40}]

    Diseñe un modelo de clases cuyo objetivo sea representar el proceso de atención de
    un cliente en una caja determinada, el cual desea comprar
    una o varias entradas para uno o varios conciertos.

\end{ProbPart}
\begin{ProbPart}[difficulty={15}]

    Una vez pensada la parte anterior, ``eleve'' su perspectiva y fíjese ahora en todo el local.
    Incorpore al modelo de la parte anterior no más de dos clases para considerar la atención de los
    clientes según su número de atención.

\end{ProbPart}

\begin{ProbPart}[difficulty={5}]

    Visualice una relación de herencia real entre dos clases de su modelo.

\end{ProbPart}

\end{Problem}


\end{spacing}
\end{document}
