# Arquitectura

## Definición de arquitectura

### Definición de arquitectura

\subsubtitleB{¿A qué nos referimos con arquitectura?}
\vspace{-0.5em}
\buildrboxx{}
Establece la estructura global de una aplicación.
\finishrboxx

- Define el marco para el diseño y el desarrollo de un Sistema de Información.
    - Es un modelo que contiene:
        - Un conjunto de componentes.
        - Y sus relaciones.
    - Provee una visión global para simplificar la comprensión de la solución modelada.
- El nivel de abstracción es más alto que el del diseño.

### Definición de arquitectura

\subsubtitleB{Características de las arquitecturas}

- Se puede refinar en sucesivas iteraciones.
    - Va mostrando cada vez \texthigh{más detalles}.
    - Así se van modelando cada uno de los módulos que componen la solución.

- Primero la arquitectura, después el diseño:
    - El nivel de abstracción de la arquitectura permite manejar de mejor manera la complejidad
    del problema.
    - Una vez que esta complejidad esté bajo control, se profundiza en el
    problema hasta llegar a la solución.

- Se enfoca en diversos aspectos del problema.
    - La estructura interna.
    - La estructura de los datos.
    - La estructura informática.

### Definición de arquitectura

\subsubtitleB{Arquitectura lógica: enfoque \itt{Top-Down}}

- Es una de las maneras más típicas de abordar un problema.
- Al principio se tiene una visión global del sistema.
    - Se lo ve como un todo unitario.
- Luego, se identifica sus componentes internos más primarios.
- Más adelante, se sigue \texthigh{profundizando} en el análisis,
caracterizando más detalladamente cada uno de esos componentes.
- Se profundiza tantas veces como sea necesario:
    - Para obtener una granularidad suficiente como para describir los
    detalles escenciales de cada componente.

### Definición de arquitectura

\subsubtitleB{Arquitectura lógica: enfoque \itt{Top-Down}}

\def\distFlow{2em}
\begin{center}\begin{tikzflowchart}
  \tikzstyle{flowElemSp} = [normalfig, rounded corners=3mm, text width=11mm, line width=1.5pt, minimum height=1.5em]
  \tikzstyle{flowSp2} = [flow2, flowElemSp, draw=black!50!blue]
  \tikzstyle{flowSp3} = [flow3, flowElemSp, draw=black!50!green]
  \tikzstyle{flowSp2in} = [flow2, flowElemSp, draw=black!50!blue]
  \tikzstyle{flowSp3in} = [flow3, flowElemSp, draw=black!50!green]
  \node (a1) [flowSp3,text width=40mm, text height=2em] {};
    \node[anchor=center] at (a1.center) {Sistema};
  \node (a2) [flowSp3, text width=80mm, text height=3em, below=\distFlow of a1] {};
    \node[anchor=north] at (a2.north) {Sistema};
    \node [flowSp2, anchor=west] at ($ (a2.west)+(1em,-0.5em) $) {Comp1};
    \node [flowSp2, anchor=center] at ($ (a2.center)+(0,-0.5em) $) {Comp2};
    \node [flowSp2, anchor=east] at ($ (a2.east)+(-1em,-0.5em) $) {Comp3};
  \node (a2a) [flowSp3, text width=34mm, text height=3em, below=\distFlow of a2] {};
    \node[anchor=north] at (a2a.north) {Sistema, Comp2};
    \node [flowSp2, anchor=west] at ($ (a2a.west)+(0.5em,-0.5em) $) {Comp2a};
    \node [flowSp2, anchor=east] at ($ (a2a.east)+(-0.5em,-0.5em) $) {Comp2b};
  \node (a2b) [flowSp3, text width=34mm, text height=3em, left=0.6em of a2a] {};
    \node[anchor=north] at (a2b.north) {Sistema, Comp1};
    \node [flowSp2, anchor=west] at ($ (a2b.west)+(0.5em,-0.5em) $) {Comp1a};
    \node [flowSp2, anchor=east] at ($ (a2b.east)+(-0.5em,-0.5em) $) {Comp1b};
  \node (a2c) [flowSp3, text width=34mm, text height=3em, right=0.6em of a2a] {};
    \node[anchor=north] at (a2c.north) {Sistema, Comp3};
    \node [flowSp2, anchor=west] at ($ (a2c.west)+(0.5em,-0.5em) $) {Comp3a};
    \node [flowSp2, anchor=east] at ($ (a2c.east)+(-0.5em,-0.5em) $) {Comp3b};
  \draw[arrow,line width=2pt] (a1) -- (a2);
  \draw[arrow,line width=2pt] (a2) -- (a2a);
  \draw[arrow,line width=2pt] (a2) -- (a2b);
  \draw[arrow,line width=2pt] (a2) -- (a2c);
\end{tikzflowchart}\end{center}

### Definición de arquitectura

\subsubtitleB{Arquitectura lógica: enfoque \itt{Top-Down}}

- ¿Cuál será el término del refinamiento?
    - Una descripción muy detalada de la solución.
    - Una descripción que caracterice las componentes más internas de cada módulo.
        - Corresponde al \texthigh{diagrama de clases}

- Vale decir:
    - En un extremo tenemos la arquitectura.
    - En el otro extremo tenemos el diseño de clases.
