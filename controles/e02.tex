\documentclass[11pt]{article}

\newcommand{\hmwkCode}{Control 2}
\newcommand{\hmwkDueDate}{S\'{a}bado 7 de noviembre de 2015}
\newcommand{\hmwkClassTime}{Tiempo total: 80 minutos. Con apuntes personales.}
\usepackage{in77j.exercise}

\setlength{\parskip}{0.7\baselineskip}
\begin{document}
\begin{spacing}{1.2}
\thispagestyle{empty}
\evaluationTitle

\newcommand{\separatorLine}{\begin{center}\rule{.6\textwidth}{1pt}\end{center}}

\begin{Problem}

    \bld{Considere el proceso de toma de horas médicas en un hospital.}

En esta evaluación vamos a analizar el proceso en que un paciente desea
reservar o anular una hora médica con un doctor, en un hospital, de
forma presencial en la misma institución.

En particular, su modelo debe cubrir lo siguiente:
\begin{enumerate}
    \item Un paciente puede tener una ficha médica en el hospital.
    \item Un paciente deseará agendar una hora médica:
    \begin{enumerate}
        \item Ya sea con un doctor en particular.
        \item O con cualquier doctor de una especialidad dada.
        \item En ambos casos hay que ofrecerle al paciente las primeras
            horas disponibles, sean cuando sean.
    \end{enumerate}
    \item Un paciente podrá ir al hospital a anular una hora médica
        previamente agendada.
    \item Un paciente podrá llegar al hospital para ser atendido:
    \begin{enumerate}
        \item Deberá pagar su bono. Suponga que lo hace en efectivo, a un
            valor que depende de cada médico, sin preocuparnos de diferentes
            valores en función de Fonasa, Isapres o atención particular.
        \item Al momento de ingresar a la oficina del médico, éste ya deberá
            tener en sus manos la ficha del paciente (si es que hay). Fíjese
            en quién debe estar a cargo de esto.
    \end{enumerate}
\end{enumerate}

\begin{ProbPart}[difficulty={40}]

Ilustre mediante un diagrama de casos de uso todo el proceso en cuestión. Tenga especial
cuidado al identificar los actores.

\end{ProbPart}

\begin{ProbPart}[difficulty={40}]

Elabore un diagrama de dominio para este proceso.

\end{ProbPart}

\begin{ProbPart}[difficulty={40}]

Elabore para cada caso de uso (de ser más de 5 casos de uso, privilegie los 5 más importantes)
un diagrama de secuencia ilustrando la interacción entre los objetos involucrados.

\end{ProbPart}

\begin{ProbPart}[difficulty={40}]

En base a todo lo realizado anteriormente, elabore un diagrama de clases que incluya:

\begin{itemize}[topsep=0pt]
    \item Todas las clases involucradas en su modelo.
    \item Todos los atributos verdaderamente relevantes.
    \item Todas las operaciones visualizadas en el diagrama de secuencia.
    \item Las relaciones relevantes para reflejar adecuadamente la realidad.
\end{itemize}

\end{ProbPart}
\end{Problem}

\begin{NewAnswer}
\end{NewAnswer}

\end{spacing}
\end{document}
