\documentclass[11pt,letterpaper]{article}

\newcommand{\hmwkCode}{Examen}
\newcommand{\hmwkDueDate}{Entrega: lunes 28 de diciembre de 2015}
\newcommand{\hmwkClassTime}{}
\usepackage{in77j.exercise}
\usepackage{mathabx}

\setlength{\parskip}{0.7\baselineskip}
\setitemize{topsep=0pt}
\begin{document}
\begin{spacing}{1.2}
\thispagestyle{empty}
\evaluationTitle

\newcommand{\separatorLine}{\begin{center}\rule{.6\textwidth}{1pt}\end{center}}

\begin{Problem}

    Un restaurant necesita modelar un sistema que le permita procesar
    los pedidos de sus clientes desde su página web. Hasta ahora solamente atendían de manera presencial, pero se
    han decidido a dar un paso adelante y ser capaces de procesar pedidos de sus clientes a través
    de internet.
    
    Estos pedidos
    están pensados para ser retirados en el local por los mismos clientes o para ser entregados a domicilio.

    Algunas observaciones:
    \begin{itemize}[itemsep=0pt,topsep=0pt]
        \item Un pedido consiste en un listado de productos, de los cuales se indica la cantidad solicitada.
        \item Junto con lo anterior, los pedidos deben indicar si el cliente
            los retirará personalmente o si prefiere recibirlos en su domicilio.
        \item Los productos son seleccionados por el cliente de un menú donde se lista, en una sola y larga
            página, todos los productos (ya sea platos de comida, bebestibles, postres, etc.).
        \item Los clientes se identifican con su nombre y su número de teléfono. Este último dato
            se usa como el identificador.
        \item El pago de los pedidos se hace siempre en efectivo y 
            cuando el cliente recibe los productos, ya sea cuando los
    va a buscar al local o cuando los recibe en su domicilio.
        \item Por ahora supondremos que ningún cliente se arrepiente de su pedido.
        \item La cocina del restaurant está ocupada no solamente atentiendo pedidos
            de internet, sino también procesando los pedidos realizados por los clientes 
            que están físicamente en los comedores del restaurant.
        \item Cada vez que un pedido es ingresado a través de internet, queda ``encolado''
            hasta que un cocinero lo toma para procesarlo. Cuando el cocinero tiene el
            pedido listo, actualiza en el sistema el estado del pedido, marcándolo como ``listo''.
        \item Si un cliente llega al local antes de que su pedido esté listo, se le invita a
            permanecer en un cómodo salón de espera, y cuando el pedido es marcado como ``listo'',
            se le cobra al cliente y se le entrega su pedido.
        \item Obviamente, si el cliente llega después de que su pedido esté listo, se le cobra
            inmediatamente y se le entrega su pedido.
        \item Para los pedidos a domicilio, éstos son repartidos una vez que han sido marcados
            como ``listos'' por un cocinero. Suponga por ahora que siempre hay repartidores
            listos para salir a terreno.
    \end{itemize}

    Considerando el problema completo, realice las siguientes tareas:

    \begin{enumerate}
        \item Establezca los requisitos funcionales y no funcionales.
        \item Realice un diagrama de casos de uso para este problema, indicando como mínimo seis
            casos de uso.
        \item Escriba la narrativa de los \bld{dos} casos de uso que usted estime como los más
            importantes de este sistema. Justifique su decisión.
        \item Identifique las clases candidatas mediante un diagrama de dominio, indicando 
            los atributos más importantes.
        \item Realice un diagrama de secuencia que ilustre el proceso en que un cliente
            hace un pedido desde la página web. Considere toda la información que el sistema necesita
            recibir de parte del cliente, y la información que este último requiere del sistema.
        \begin{itemize}
            \item Determine bien en este diagrama las interfaces, las entidades y el o los controladores.
        \end{itemize}
        \item Realice el diagrama de clases completo.
    \end{enumerate}

\end{Problem}

\end{spacing}
\end{document}
