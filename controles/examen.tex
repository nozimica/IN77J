\documentclass[11pt,letterpaper]{article}

\newcommand{\hmwkCode}{Examen}
\newcommand{\hmwkDueDate}{Entrega: miércoles 30 de diciembre de 2015}
\newcommand{\hmwkClassTime}{}
\usepackage{in77j.exercise}
\usepackage{mathabx}

\setlength{\parskip}{0.7\baselineskip}
\setitemize{topsep=0pt,itemsep=1pt}
\begin{document}
\begin{spacing}{1.2}
\thispagestyle{empty}
\evaluationTitle

\newcommand{\separatorLine}{\begin{center}\rule{.6\textwidth}{1pt}\end{center}}

\begin{Problem}

    Los dueños de un restaurant que hasta ahora solamente atendía de manera presencial, se
    han decidido a dar un paso adelante y quieren procesar pedidos de sus clientes a través
    de internet.  Estos pedidos
    están pensados para ser retirados en el local por los mismos clientes o para ser entregados a domicilio.

    Algunas observaciones:
    \begin{itemize}
        \item Un pedido consiste en un listado de productos, de los cuales se indica la cantidad solicitada de cada uno.
        \item Junto con lo anterior, los pedidos deben indicar si el cliente
            los retirará personalmente o si prefiere recibirlos en su domicilio.
        \item Los productos son seleccionados por el cliente a partir de un menú donde se lista, en una sola y larga
            página, todos los productos (ya sea platos de comida, bebestibles, postres, etc.).
        \begin{itemize}
            \item Las bebidas solamente indican su nombre.
            \item Los platos de comida, \itt{sandwichs} y postres deben indicar también sus ingredientes.
        \end{itemize}
        \item A los clientes se les pide su nombre, dirección y número de teléfono. Este último dato
            se usa como identificador.
        \item Se debe reconocer a los clientes que ya han hecho un pedido antes, para ofrecerles promociones.
        \begin{itemize}
            \item Cada pedido del cliente le permite acumular 1 punto por cada \$1.000 de compra.
        \end{itemize}
        \item El pago de los pedidos se hace siempre en efectivo y 
            después de que el cliente reciba los productos, ya sea cuando los
    va a buscar al local o cuando los recibe en su domicilio.
        \item Por ahora supondremos que ningún cliente se arrepiente de su pedido.
        \item La cocina del restaurant está ocupada no solamente atentiendo pedidos
            de internet, sino también procesando los pedidos realizados por los clientes 
            que están físicamente en los comedores del restaurant.
        \item Cada vez que un pedido se ingresa a través de internet, queda ``encolado''
            hasta que un cocinero lo toma para procesarlo.
        \begin{itemize}
            \item Suponga que el cocinero hace todo el proceso, tanto de preparar el plato
                como de recolectar las bebidas y postres, dejando todo dentro de una caja para que
                el cliente se la lleve a su casa o para que le sea llevada a su domicilio.
            \item  El sistema debe proveer una interfaz para ser usada por los cocineros, a través de la cual
                vean los detalles del pedido y lo marquen como ``procesando''.
            \item Cuando el cocinero tenga el
            pedido listo, actualizará en el sistema el estado del pedido marcándolo como ``listo''.
        \end{itemize}
        \item Si un cliente llega al local antes de que su pedido esté listo, se le invita a
            permanecer en un cómodo salón de espera, y cuando el pedido sea marcado como ``listo'',
            se le cobrará al cliente y se le entregará su pedido.
        \item Obviamente, si el cliente llega después de que su pedido esté listo, se le cobra
            inmediatamente y se le entrega su pedido.
        \item Los pedidos a domicilio son repartidos una vez que hayan sido marcados
            como ``listos'' por un cocinero. Suponga por ahora que siempre hay repartidores
            listos para salir a terreno.
        \item Es necesario que para todo envío a domicilio quede registrado el nombre del repartidor que
            lo llevó a cabo.
    \end{itemize}

    \vspace{1em}
    Considerando el problema completo, realice las siguientes tareas:

    \begin{enumerate}
        \item Establezca los requisitos funcionales y no funcionales.
        \item Realice un diagrama de casos de uso para este problema, indicando como mínimo seis
            casos de uso.
        \item Escriba la narrativa de los \bld{dos} casos de uso que usted estime como los más
            importantes de este sistema. Justifique su decisión.
        % \item Identifique las clases candidatas mediante un diagrama de dominio, indicando 
        %     los atributos más importantes.
        \item Realice un diagrama de secuencia que ilustre el proceso en que un cliente
            hace un pedido desde la página web. Considere toda la información que el sistema necesita
            recibir de parte del cliente, y la información que este último requiere del sistema.
        \begin{itemize}
            \item Determine bien en este diagrama las interfaces, las entidades y el o los controladores.
            \item Considere este proceso desde el momento en que el cliente visita la página para hacer
                el pedido, hasta que este último está ya ingresado en el sistema.
            \item Explique qué cambiaría usted en este diagrama para proveer dos páginas web distintas,
                una para \itt{notebooks} y computadores de escritorio; y otra para dispositivos móviles.
        \end{itemize}
        \item Realice el diagrama de clases completo, el cual debe ser capaz de representar todas
            las instancias involucradas en este problema.
        \begin{itemize}
            \item No olvide pensar en cómo está estructurado un pedido.
            \item Incorpore en este modelo adecuadamente las características ---mencionadas en la descripción--- de los productos ofrecidos.
        \end{itemize}
    \end{enumerate}

\end{Problem}

\end{spacing}
\end{document}
