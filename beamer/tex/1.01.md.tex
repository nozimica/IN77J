# Introducción

## Presentación del curso

### Presentaciones

* \subsubtitle{Estudiantes}
    - \subsubtitleC{Conocimientos previos.}
    - \subsubtitleC{Experiencia profesional.}
    \pause
* \subsubtitle{Profesor:}
    \pause

    \vspace{-6mm}
    \begin{center}
    \hspace{20mm}\begin{customRoundedBox}{}
    \centering
        \subsubtitle{Nicolás Eugenio Ozimica}\\
        \vspace{4mm}
        \subsubtitleC{Ingeniero Civil en Computación}\\
        \subsubtitleC{Universidad de Chile}
    \end{customRoundedBox}
    
    \hspace{60mm}\begin{customRoundedBox}{}
    \centering
        \subsubtitleC{nozimica@dcc.uchile.cl}\\
        \subsubtitleC{nozimica@gmail.com}
    \end{customRoundedBox}
    \end{center}

### Objetivos del Curso

- Dominar el paradigma de programación orientada a objetos, desde un punto de vista 
  práctico, apoyado en un adecuado marco teórico.
  \pause
- Modelar una arquitectura de \itt{e-business} utilizando orientación a objetos
  con el objetivo de apoyar los procesos de negocio
  de una empresa u organización.
  \pause
- Como herramienta en la ilustración de este modelo se utilizará los diagramas UML.
  \pause
- Analizar el modelo de datos subyacente.
  \pause
- Lo teórico sólo se domina a través de la práctica: 
    - Implementar algunos de nuestros diseños en un lenguaje de programación real: Java o Python.

### Actividades

- Clases teóricas los días sábado:
    - Horario: 8:30 a 11:45.
    - Horario alternativo: 9:00 a 12:15.

- Clases auxiliares:
    - Horario por definir.

- Evaluaciones:
    - Controles, tareas y proyectos:
        - Cantidad y fechas por definir.
    - Examen individual.
        - Última clase de cátedra.


### Contenido del curso

1. Conceptos básicos de la Orientación a Objetos (OO).
1. Arquitectura tecnológica y de software.
1. Requisitos de software.
1. Modelamiento orientado a objetos con UML.
    - Diagramas de Casos de Uso.
    - Diagramas de Secuencias.
    - Modelo de clases, objetos y componentes.
    - Modelamiento de los datos.
1. Modelamiento de la arquitectura de software.
    - Modelo de componentes del e-business.
    - Modelos UML para entrega y distribución.
1. Relación con la arquitectura de un *e-business*.
