# Requerimientos de Software

## Ejemplo introductorio: Juego de Dados

### Requerimientos de Software

\subsubtitleB{Ejemplo introductorio: Juego de Dados}\newline

Para introducir los siguientes conceptos, nos apoyaremos en un ejemplo simple pero muy útil:
\textcolor{blue}{El Juego de Dados}.\newline

Actividades a realizar:

- Especificación de requerimientos.
- Modelado del dominio.
- Definición de las interacciones.
- Definición de la estructura.
- Ejemplo de codificación.


## Especificación de requerimientos

### Especificación de requerimientos

\subsubtitleB{Requerimientos:}\newline

Corresponde a una especificación muy concreta de lo que se espera que lleve a cabo el
\itt{software} que estamos analizando.\newline

En nuestro caso:

- Se requiere de una aplicación que le permita a un jugador ``lanzar'' dos dados, obteniendo
siempre el valor de cada uno de ellos.


### Especificación de requerimientos

\subsubtitleB{Clasificación de requerimientos:}\newline

Existen fundamentalmente dos tipos:

\begin{rboxx}{}
\begin{description}
    \item[Funcionales:] Los que motivan el problema a resolver.
\end{description}
\end{rboxx}

\begin{rboxx}{}
\begin{description}
    \item[No funcionales:] Los que acompañan a los req. funcionales.
\end{description}
\end{rboxx}

### Especificación de requerimientos

\subsubtitleB{Requerimientos Funcionales:}\newline

- Describen todas las cosas que debe hacer el sistema.
    - Describen el \textcolor{blue}{¿Qué?}.
    - No el \textcolor{blue}{¿Cómo?}.
- En un principio son solamente requerimientos.
- Al avanzar el proyecto se convierten en:
    - Los algoritmos
    - Las lógicas
    - El código del \itt{software}.

### Especificación de requerimientos

\subsubtitleB{Requerimientos No Funcionales:}\newline

Si bien no describen el problema propiamente tal, sí describen características anexas a éste y
se suelen agrupar en:

- Requerimientos de interfaz
- Requerimientos ergonómicos
- Requerimientos de desempeño
- Requerimientos de disponibilidad
- Requerimientos de entorno
- Requerimientos de entrenamiento

### Especificación de requerimientos

\subsubtitleB{Requerimientos de interfaz:}\newline

Una interfaz es la forma en que interactúan dos sistemas.\newline

- En nuestro caso, es la forma en que se requiere que interactúe nuestro sistema:
    - Con sus usuarios
    - Con otros sistemas
- Corresponde a la especificación formal de los datos que el sistema recibe o envía hacia
el exterior.
- Usualmente lo que se especifica es:
    - El protocolo
    - El tipo de información
    - El medio de comunicación
    - El formato de los datos comunicados

### Especificación de requerimientos

\subsubtitleB{Requerimientos ergonómicos:}\newline

- Son fundamentalmente el opuesto a los requerimientos de interfaz:
    - Corresponden a cómo deben interactuar los usuarios con el sistema.

- Lo típico es especificar la ``GUI'': \itt{Graphic User Interface}.

- La ergonomía va en cómo el sistema satisface y se adapta a las necesidades
de sus usuarios.

### Especificación de requerimientos

\subsubtitleB{Requerimientos de desempeño:}\newline

- Especifican el desempeño esperado de parte del sistema.
    - Rapidez
    - Transacciones a procesar
    - Recursos a consumir

- Especialmente importantes en sistemas que procesan datos en tiempo real.
    - En estos casos el desempeño es crítico.
    - Hay operaciones que no pueden demorarse ``tanto'' como uno o dos segundos.

### Especificación de requerimientos

\subsubtitleB{Requerimientos de disponibilidad:}\newline

- Especifican cuán disponible ha de estar un sistema para sus usuarios (personas u
otros sistemas).
    - Tiempo mínimo de operación.
    - Portabilidad: instalable en varios equipos o sistemas operativos.
    - Flexibilidad: que funcione en distintas plataformas: celulares, tablets, etc.
    - Actualizaciones: que estén disponibles oportunamente.

- También son importantes en sistemas que procesan datos en tiempo real.
    - Sistemas como éstos no pueden estar ``caídos'' mucho tiempo.

### Especificación de requerimientos

\subsubtitleB{Requerimientos de entorno:}\newline

- Se refiere a los requerimientos asociados al entorno dentro del cual funcionará el sistema:
    - Sistemas operativos
    - Bases de datos

- El entorno puede implicar peligros que amenacen el funcionamiento del sistema:
    - Se requerirá de determinado grado de robustez para tolerar adecuadamente estos problemas.
    - Peligros tales como:
        - Congestión de usuarios.
        - Congestión de dispositivos.
        - Errores en los datos recibidos.

### Especificación de requerimientos

\subsubtitleB{Requerimientos de entrenamiento:}\newline

- Requerimientos enfocados en quienes utilizarán el sistema.
    - Qué tipos de usuarios y/u operadores.
    - Listado de los manuales necesarios.
    - Idiomas de los manuales.

- La mayoría de las veces estos requerimientos no se ven reflejados en la implementación
del sistema propiamente tal, sino que repercuten en la entrega de manuales o incluso en
la organización de sesiones de entrenamiento para los futuros operadores.

## Definición de las interacciones.

### Definición de las interacciones.

Para especificar más claramente los requerimientos de un sistema, es bueno concentrarse en
las interacciones que se espera que ocurran una vez que éste esté operando.\newline

Estas interacciones se suelen ilustrar mediante:

\begin{rboxx}{}
    \textcolor{blue}{Diagramas de Casos de Uso.}
\end{rboxx}

## Diagramas de Casos de Uso

### Diagramas de Casos de Uso

\subsubtitleB{Introducción}\newline

- Es un tipo de diagrama que mediante símbolos específicos, documentan la manera en que los
actores interactúan con el sistema, y viceversa.

- Establecen las funcionalidades necesarias para que los usuarios interactúen exitosamente con el sistema.

\begin{rboxx}{}
    Son un ``índice'' de las funcionalidades del sistema,
    asociadas a los actores que las utilizan.
\end{rboxx}

- Su objetivo es especificar las distintas interacciones entre el sistema y sus usuarios, los que
están siempre en el mundo exterior.

### Diagramas de Casos de Uso

\subsubtitleB{Ejemplo:} funcionalidades de un cajero automático\newline

\exA

### Diagramas de Casos de Uso

\subsubtitleB{Elementos: casos de uso}\newline

\columnsbegin

\column{.3\textwidth}

\centering\scalebox{0.9}{%
\begin{tikzpicture}%
    \umlusecase{Cargar BIP!}
\end{tikzpicture}%
}

\column{.1\textwidth}
\column{.6\textwidth}
\bld{Caso de uso:} Identifica una funcionalidad que el sistema
debe ofrecer a sus usuarios o una meta que debe alcanzar.\newline

\columnsend

- Se describen con una frase de caracter verbal.
- Su nombre debe ser lo más breve posible.
- Identifican una funcionalidad concreta del sistema.
- Están centrados en la meta, no en el proceso.


### Diagramas de Casos de Uso

\subsubtitleB{Elementos: estereotipos}\newline

\columnsbegin

\column{.3\textwidth}

\centering\scalebox{0.6}{%
\begin{tikzpicture}%
    \umlusecase[name=Sacar]{Sacar dinero}
    \umlusecase[x=0,y=-4,name=Valida]{Valida usuario}
    \umlinclude{Sacar}{Valida}
\end{tikzpicture}%
}

\column{.1\textwidth}
\column{.6\textwidth}
\bld{Estereotipos:} Se usan para indicar que un caso de uso \textcolor{blue}{incluye}
la lógica contenida en otro caso de uso.

- Por lo general son casos de uso incluidos por más de un caso de uso.
- No necesariamente están asociados a actor alguno.

Existe un estereotipo llamado ``extend'', el cual no es muy conveniente de usar.

\columnsend

### Diagramas de Casos de Uso

\subsubtitleB{Elementos: actores}\newline

\columnsbegin

\column{.3\textwidth}

\vspace{-5mm}
\centering\scalebox{0.9}{%
\begin{tikzpicture}%
    \umlactor{Usuario}
\end{tikzpicture}%
}

\column{.1\textwidth}
\column{.6\textwidth}
\bld{Actor:} Más que un actor, corresponde a un \textcolor{blue}{rol} que es interpretado por una entidad
externa con respecto al sistema.\newline
\columnsend

- Puede ser una persona.
- Puede ser otro sistema.
- Incluso puede corresponder a un dispositivo.
- Son los \textcolor{blue}{sujetos} en las oraciones que describen cómo es utilizado un sistema.


### Diagramas de Casos de Uso

\subsubtitleB{Elementos: sub-actores}\newline

\columnsbegin

\column{.3\textwidth}

\centering\scalebox{0.7}{%
\begin{tikzpicture}%
    \umlactor{Usuario}
    \umlactor[x=0,y=-4]{Cliente}
    \umlinherit{Cliente}{Usuario}
\end{tikzpicture}%
}

\column{.1\textwidth}
\column{.6\textwidth}
\bld{Sub-actor:} Corresponde a un actor que es una \textcolor{blue}{especialización} de
otro actor, a quien apunta la flecha.

- El sub-actor está relacionado a sus propios casos de uso.
- Y también está relacionado a los casos de uso del actor a quien especializa.
- Es una especie de ``herencia'', pero en roles.

\columnsend

### Diagramas de Casos de Uso

\subsubtitleB{Elementos: asociaciones}\newline

\columnsbegin

\column{.3\textwidth}

\vspace{-5mm}
\centering\scalebox{0.6}{%
\begin{tikzpicture}%
    \umlactor{Cliente}
    \umlusecase[x=4,name=Deposita]{Deposita sobre}
    \umlassoc{Cliente}{Deposita}
\end{tikzpicture}%
}

\column{.1\textwidth}
\column{.6\textwidth}
\bld{Asociación:} Indica la relación entre un actor y un caso de uso.\newline

\columnsend

- Se representa mediante una línea contínua que conecta un actor a un caso de uso.
- No se suele indicar dirección.
- De hacerlo, hay que indicar qué se quiere expresar con ello:
    - Flujo de información.
    - Si el actor es el que invoca al caso de uso o al revés.

### Diagramas de Casos de Uso

\subsubtitleB{Elementos: sistema}\newline

\columnsbegin

\column{.3\textwidth}

\centering\scalebox{0.5}{%
\begin{tikzpicture}%
    \begin{umlsystem}[x=4]{Cajero automático}%
    \end{umlsystem}%
\end{tikzpicture}%
}

\column{.1\textwidth}
\column{.6\textwidth}
\bld{Sistema:} Establece el límite del sistema con respecto a los actores:

- En su interior, los casos de uso.
- En su exterior, los actores.

Deja en claro que el sistema modelado corresponde a los usos llevados a cabo por actores externos.

\columnsend

### Diagramas de Casos de Uso

\subsubtitleB{Características}\newline

- Los casos de uso especifican los requerimientos del problema modelado.
- Especifican \textcolor{blue}{lo que debe hacer} el sistema.
- No especifican cómo esas metas serán logradas.
- Están centrados en lo que el sistema ofrece a los actores que usarán el sistema.
- El sistema se considera más como un conjunto de requerimientos que como una solución.

### Diagramas de Casos de Uso

\subsubtitleB{Cómo diseñarlos}\newline

- ¿Qué casos de uso incluiremos dentro del sistema?
    - Los casos de uso conforman las características del sistema que se está modelando.
    - Se presentan los casos de uso visibles y significativos para los actores.
    - Si se requiere incluir muchos casos de uso, por lo general es mejor separarlos en dos diagramas.
    - Un sistema con muchos casos de uso puede tratarse en realidad de dos sistemas que estamos mezclando.
- ¿Qué actores son relevantes para esos casos de uso?

