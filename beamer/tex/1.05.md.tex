## Narrativa de Casos de Uso

### Narrativa de Casos de Uso

- Los casos de uso no solamente se presentan en la forma de diagramas.
- Una vez que se tiene claro los actores y cómo éstos interactúan con el sistema, se hace
necesario ahondar en esas interacciones:

\begin{rboxx}{}
    \textcolor{blue}{Hay que detallar cada Caso de Uso.}
\end{rboxx}

- Esto se consigue a través de la narrativa de cada uno de ellos.

### Narrativa de Casos de Uso

\subsubtitleB{Características}\newline

- Una narrativa de un caso de uso es una descripción de la historia completa de lo que hacen los actores
a medida que llevan a cabo un caso de uso.
    - Esto termina cuando los actores involucrados logran su objetivo.
- Corresponden a la herramienta más aplicada en la actualidad para especificar los requerimientos funcionales.
- Dado que son una descripción textual, no se requiere de mayores conocimientos técnicos para comprenderlos.

### Narrativa de Casos de Uso

\subsubtitleB{Cómo escribirlos}\newline

- Primero se identifica a los \textcolor{blue}{actores}.
- Luego se identifican las necesidades de esos actores.
    - Esto permite visualizar los casos de uso que ellos requieren.
    - Se detectan tanto las necesidades de los actores como sus eventuales objetivos.
- Estas necesidades dan origen a los casos de uso.
- No es necesario detectar todos los casos de uso al primer análisis. Éstos van visualizándose
a medida que se avanza en el desarrollo del proyecto.
    - Primero se especifica los casos de uso más importantes.
    - Se avanza con los otros de manera incremental.

### Narrativa de Casos de Uso

\subsubtitleB{Cómo escribirlos}\newline

- A medida que se afinan las características de cada caso de uso, se pasa por los siguientes
niveles:

\begin{rboxx}[100mm]{}
\begin{description}
    \item[Identificado o detectado:] sólo se tiene su nombre y los actores asociados.
\end{description}
\end{rboxx}

\begin{rboxx}[100mm]{}
\begin{description}
    \item[Especificado en alto nivel:] se incorpora una sinopsis.
\end{description}
\end{rboxx}

\begin{rboxx}[100mm]{}
\begin{description}
    \item[Especificado de forma extendida:] se incorpora su historia de uso y sus
    eventuales variantes.
\end{description}
\end{rboxx}


### Ejemplo de Narrativa de Casos de Uso

\vspace{-1em}
\begin{tiny}
\begin{usecase}

\addtitle{Caso de uso \#}{Hacer pedido} 

%Scope: the system under design
\addfield{Alcance:}{Atención a público}

%Level: "user-goal" or "subfunction"
% \addfield{Level:}{User-goal}

%Primary Actor: Calls on the system to deliver its services.
\addfield{Actor principal:}{Cliente}
% \addfield{Actores secund:}{}

\addfield{Sinopsis:}{El cliente deberá tener la posibilidad de realizar un pedido, entregando
todos los datos necesarios para poder ser procesado.}

\addfield{Precondiciones:}{Se le presenta al cliente el menú disponible.}
\addfield{Postcondiciones:}{\bld{Garantía}: El pedido queda registrado si el pedido es completado.}

\addscenario{Escenario principal:}{
    \item Usuario \underline{revisa el catálogo e indica sus opciones}.
    \item Se le presenta el monto total de la compra.
    \item Usuario indica forma de pago.
    \item Se procesa la forma de pago.
    \item Usuario recibe comprobante.
}

\addscenario{Escenario alternativo:}{
    \item[5.a] Pago rechazado.
}

\end{usecase}
\end{tiny}

### Elementos en una Narrativa

\subsubtitleB{Número y nombre:}\newline

- Indican el nombre del caso de uso, y también se suele indicar un número, para maś fácil referencia.

\subsubtitleB{Alcance:}\newline

- Corresponde al entorno en el cual se desenvuelve este caso de uso dentro del sistema.
- En sistemas pequeños, este elemento se puede omitir.

### Elementos en una Narrativa

\subsubtitleB{Actores principales y secundarios:}\newline

- Es el listado de los actores principales involucrados en este caso de uso.
    - Los actores principales son aquellos cuyos objetivos son satisfechos directamente
    por el caso de uso.
    - No necesariamente son quienes inician el caso de uso.
- Opcionalmente, se puede segmentar algunos actores como secundarios.
    - Si no se incluyen actores secundarios, simplemente omitir esa línea.


### Elementos en una Narrativa

\subsubtitleB{Sinopsis o resumen:}\newline

- Es simplemente el resumen de lo que hace este caso de uso.
- Es importante que sea breve.
    - Resúmenes muy largos no los leen todas las personas involucradas.
    - El detalle del caso de uso va realmente en el ``escenario principal''.

### Elementos en una Narrativa

\subsubtitleB{Precondiciones:}\newline

- Son las condiciones necesarias para que el caso de uso se lleve a cabo.
    - Si no son satisfechas, el caso de uso no se ejecuta.
- Deben ser reglas conocidas por todos los actores.
- En etapas posteriores del proyecto, estas precondiciones son útiles para
saber qué cosas no hay que revisar dentro del presente módulo.

### Elementos en una Narrativa

\subsubtitleB{Postcondiciones:}\newline

- Describen las principales consecuencias ---para el sistema--- luego de que el caso de uso haya
concluido con éxito.
- Se suele incorporar la palabra \textcolor{blue}{Garantizar}, para hacer énfasis en que
esas condiciones deben estar garantizadas si el caso concluyó correctamente.
    - También se puede indicar garantías mínimas, que son las cubiertas en todos los escenarios.


### Elementos en una Narrativa

\subsubtitleB{Escenario principal:}\newline

- Es el elemento más importante dentro del caso de uso.
- Corresponde a la descripción \textcolor{blue}{paso a paso} de cómo se lleva la interacción
entre el o los actores y el sistema.
- Cada paso debe estar descrito de la manera más concisa posible.
- Debe quedar claro quién está ejecutando cada paso.
- Nunca perder de vista que nos enfocamos en el \textcolor{blue}{qué}, no en el \textcolor{blue}{cómo}.
- Denominado también como ``escenario típico'' o escenario ideal.

### Elementos en una Narrativa

\subsubtitleB{Escenario principal:}\newline

- Alguno de estos pasos puede corresponder a su vez a otro caso de uso.
    - En este caso, estamos frente a lo que ya definimos como \itt{include}.
    - Para indicar que este es el caso, se puede subrayar el paso correspondiente.


### Elementos en una Narrativa

\subsubtitleB{Escenario alternativo:}\newline

- En algunos casos es útil identificar algunos pasos ---dentro del escenario principal--- que
pueden tener un resultado distinto a aquél que se considera como esperable.
- Se mencionan indicando el número del paso correspondiente, adjuntando una letra correlativa.
- Denominado también como ``extensión''.

