\documentclass[11pt,letterpaper]{article}

\newcommand{\hmwkCode}{Control 2}
\newcommand{\hmwkDueDate}{S\'{a}bado 7 de noviembre de 2015}
\newcommand{\hmwkClassTime}{Tiempo total: 60 minutos. Con apuntes personales.}
\usepackage{in77j.exercise}

\setlength{\parskip}{0.7\baselineskip}
\begin{document}
\begin{spacing}{1.2}
\thispagestyle{empty}
\evaluationTitle

\newcommand{\separatorLine}{\begin{center}\rule{.6\textwidth}{1pt}\end{center}}

\begin{Problem}

    \bld{Considere el proceso de toma de horas médicas en un hospital.}

En esta evaluación vamos a analizar el proceso en que un paciente desea
reservar o anular una hora médica con un doctor, en un hospital, de
forma presencial en la misma institución.

En particular, su modelo debe cubrir lo siguiente:
\begin{enumerate}
    \item Un paciente puede tener una ficha médica en el hospital.
    \item Un paciente deseará agendar una hora médica:
    \begin{enumerate}
        \item Ya sea con un doctor en particular.
        \item O con cualquier doctor de una especialidad dada.
        \item En ambos casos hay que ofrecerle al paciente las primeras
            horas disponibles, sean cuando sean.
    \end{enumerate}
    \item Un paciente podrá ir al hospital a anular una hora médica
        previamente agendada.
    \item Un paciente podrá llegar al hospital para ser atendido:
    \begin{enumerate}
        \item Deberá pagar su bono. Suponga que lo hace en efectivo, a un
            valor que depende de cada médico, sin preocuparnos de diferentes
            valores en función de Fonasa, Isapres o atención particular.
        \item Al momento de ingresar a la oficina del médico, éste ya deberá
            tener en sus manos la ficha del paciente (si es que hay). Fíjese
            en quién debe estar a cargo de esto.
    \end{enumerate}
\end{enumerate}

\begin{ProbPart}[difficulty={10}]

Ilustre mediante un diagrama de casos de uso todo el proceso en cuestión. Tenga especial
cuidado al identificar los actores.

\end{ProbPart}

\begin{ProbPart}[difficulty={30}]

Para los 4 casos de uso más importantes, elabore la narrativa correspondiente.

\end{ProbPart}

\begin{ProbPart}[difficulty={30}]

Para cada una de las narrativas anteriores, haga el diagrama de secuencia ilustrando la interacción entre los objetos involucrados.

\end{ProbPart}

\end{Problem}

\begin{NewAnswer}

\begin{ProbPart}[difficulty={10}]

El caso de uso es:

\begin{tikzpicture}%
    \umlactor[x=-3,y=-2]{Paciente}
    \umlactor[x=13,y=-3]{Funcionario}
    \umlactor[x=13,y=-10]{Médico}
    \begin{umlsystem}[x=2]{Hospital}
        \umlusecase[x=3,y=3,name=anular]{Anular hora médica}
        \umlusecase[x=3,y=1,name=agendarMedico,width=7em]{Agenda hora\\con médico}
        \umlusecase[y=-1.5,name=calendDoc,width=8em]{Obtiene agenda\\del médico}
        \umlusecase[x=3,y=-6,name=agendarEspec,width=7em]{Agenda hora\\por especialidad}
        \umlusecase[x=6,y=-3.5,name=listadoDocs,width=8em]{Obtiene médicos\\por especialidad}
        \umlusecase[x=3,y=-8,name=atenderse]{Atenderse}
        \umlusecase[x=0,y=-10,name=obtFicha,width=6em]{Buscar ficha\\paciente}
        \umlusecase[x=3,y=-12,name=pagarAtencion,width=6em]{Pagar\\bono}
        \umlusecase[x=6,y=-10,name=verDoc,width=6em]{Ver al\\médico}
        \umlinclude{agendarMedico}{calendDoc}
        \umlinclude{agendarEspec}{listadoDocs}
        \umlinclude{listadoDocs}{calendDoc}
        \umlinclude{atenderse}{obtFicha}
        \umlinclude{atenderse}{pagarAtencion}
        \umlinclude{atenderse}{verDoc}
    \end{umlsystem}
    \umlassoc{Paciente}{agendarMedico}
    \umlassoc{Funcionario}{agendarMedico}
    \umlassoc{Paciente}{agendarEspec}
    \umlassoc{Funcionario}{agendarEspec}
    \umlassoc{Paciente}{anular}
    \umlassoc{Funcionario}{anular}
    \umlassoc{Paciente}{atenderse}
    \umlassoc{Funcionario}{atenderse}
    \umlassoc{Médico}{verDoc}
\end{tikzpicture}%

\end{ProbPart}

\newpage

\begin{ProbPart}[difficulty={30}]

    He aquí cuatro ejemplos de narrativas:

\begin{nzusecase}
\nzucheader{Agendar hora con médico}
\nzucitem{Alcance}{Módulo principal}
\nzucitem{Actor principal}{Paciente, Funcionario}
\nzucitem{Sinopsis}{El paciente solicita una hora para atenderse con un médico en particular, siempre y cuando haya horas disponibles de atención con aquel profesional.}
\nzucitem{Precondiciones}{El paciente sabe el nombre del médico. El paciente está físicamente en el hospital.}
\nzucitem{Postcondiciones}{Garantía: El paciente tendrá su hora agendada.}
\nzucitembig{Escenario principal}{%
    \item Paciente indica el nombre del médico de su preferencia.
    \item Funcionario obtiene la agenda de ese médico.
    \item Funcionario le presenta al paciente las próximas horas disponibles.
    \item Paciente selecciona una de esas horas.
    \item Funcionario marca en la agenda del médico esa hora como tomada.
    \item Funcionario le confirma al paciente la hora tomada.
}
\nzucitembig{Escenarios alternativos}{%
\item[3a.] No hay horas disponibles.
}
\nzucitembig{Escenarios alternativos}{%
\item[4a.] Paciente no le sirve o no puede asistir a ninguna de las horas disponibles.
}
\end{nzusecase}

\begin{nzusecase}
\nzucheader{Agendar hora por especialidad}
\nzucitem{Alcance}{Módulo principal}
\nzucitem{Actor principal}{Paciente, Funcionario}
\nzucitem{Sinopsis}{El paciente solicita una hora para atenderse con cualquier médico disponible de una especialidad en particular.}
\nzucitem{Precondiciones}{El paciente sabe el nombre de la especialidad. El paciente está físicamente en el hospital.}
\nzucitem{Postcondiciones}{Garantía: El paciente tendrá su hora agendada.}
\nzucitembig{Escenario principal}{%
    \item Paciente indica la especialidad en la que necesita atenderse.
    \item Funcionario obtiene la agenda combinada de todos los médicos para la especialidad solicitada por el paciente.
    \item Funcionario le presenta al paciente las próximas horas disponibles.
    \item Paciente selecciona una de esas horas.
    \item Funcionario marca en la agenda del médico esa hora como tomada.
    \item Funcionario le confirma al paciente la hora tomada.
}
\nzucitembig{Escenarios alternativos}{%
\item[3a.] No hay horas disponibles.
}
\nzucitembig{Escenarios alternativos}{%
\item[4a.] Paciente no le sirve o no puede asistir a ninguna de las horas disponibles.
}
\end{nzusecase}

\begin{nzusecase}
\nzucheader{Anular hora con médico}
\nzucitem{Alcance}{Módulo principal}
\nzucitem{Actor principal}{Paciente, Funcionario}
\nzucitem{Sinopsis}{El paciente solicita anular una hora que tenía previamente agendada con algún médico en el hospital.}
\nzucitem{Precondiciones}{El paciente sabe el nombre del médico. El paciente está físicamente en el hospital.}
\nzucitem{Postcondiciones}{Garantía: El paciente anulará su hora. La hora quedará disponible para otro paciente.}
\nzucitembig{Escenario principal}{%
    \item Paciente indica el médico y la hora en que tenía agendada una atención.
    \item Funcionario obtiene la agenda de ese médico.
    \item Funcionario marca en la agenda del médico esa hora como disponible.
    \item Funcionario le confirma al paciente la hora anulada.
}
\end{nzusecase}

\begin{nzusecase}
\nzucheader{Atenderse}
\nzucitem{Alcance}{Módulo principal}
\nzucitem{Actor principal}{Paciente, Funcionario}
\nzucitem{Sinopsis}{El paciente va al hospital a ser atendido por el médico.}
\nzucitem{Precondiciones}{El paciente tiene una hora agendada con el médico. El paciente está físicamente en el hospital.}
\nzucitem{Postcondiciones}{El paciente sale del hospital ya atendido y dependiendo del caso, con un diagnóstico.}
\nzucitembig{Escenario principal}{%
    \item Paciente indica el médico y la hora que tiene agendada.
    \item Funcionario verifica que el paciente tenga la hora agendada.
    \item Funcionario indica el valor del bono a cancelar.
    \item Paciente paga el bono.
    \item Funcionario se encarga de solicitar que se envíe la ficha del paciente a la oficina del médico.
    \item Paciente espera a ser atendido por el médico.
    \item Médico atiende a paciente.
}
\nzucitembig{Escenarios alternativos}{%
\item[2a.] Paciente no tiene hora agendada.
}
\nzucitembig{Escenarios alternativos}{%
\item[4a.] Paciente no puede cancelar bono.
}
\end{nzusecase}

\end{ProbPart}

\newpage

\begin{ProbPart}[difficulty={30}]

\vspace{1em}
\bld{Agendar hora con médico:}

\tikzinlinec[0.9]{%
    \begin{umlseqdiag}
    \umlactor[no ddots]{Paciente}
    \umlactor[no ddots]{Funcionario}
    \umlboundary[no ddots]{Sistema}
    \umlcontrol[no ddots]{Controlador}
    \umlentity[no ddots]{Médicos}
    % \umlbasicobject{AgendaMédicos}
    \begin{umlcall}[op=solicitaHora(médico)]{Paciente}{Funcionario}
        \begin{umlcall}[op=consultaHoras(médico)]{Funcionario}{Sistema}
            \begin{umlcall}[op=consultaHoras(médico),return=horasMédico]{Sistema}{Controlador}
                \begin{umlcall}[op=consultaHoras(médico),return=horasMédico]{Controlador}{Médicos}
                \end{umlcall}
            \end{umlcall}
        \end{umlcall}
    \end{umlcall}
    \begin{umlcall}[op=muestraHoras()]{Funcionario}{Paciente}
    \end{umlcall}
    \begin{umlcall}[op=seleccionaHora()]{Paciente}{Funcionario}
        \begin{umlcall}[op=agendar({hora,médico})]{Funcionario}{Sistema}
            \begin{umlcall}[op=agendar({hora,médico}),return=éxito]{Sistema}{Controlador}
                \begin{umlcall}[op=agendar({hora,médico}),return=éxito]{Controlador}{Médicos}
                \end{umlcall}
            \end{umlcall}
        \end{umlcall}
    \end{umlcall}
    \begin{umlcall}[op=confirmaHora()]{Funcionario}{Paciente}
    \end{umlcall}
    \end{umlseqdiag}
}


\vspace{1em}
\bld{Agendar hora por especialidad}

\tikzinlinec[0.9]{%
    \begin{umlseqdiag}
    \umlactor[no ddots]{Paciente}
    \umlactor[no ddots]{Funcionario}
    \umlboundary[no ddots]{Sistema}
    \umlcontrol[no ddots]{Controlador}
    \umlentity[no ddots]{Médicos}
    % \umlbasicobject{AgendaMédicos}
    \begin{umlcall}[op=solicitaHora(espec)]{Paciente}{Funcionario}
        \begin{umlcall}[op=consultaHoras(espec)]{Funcionario}{Sistema}
            \begin{umlcall}[op=consultaHoras(espec),return=horasMédicos]{Sistema}{Controlador}
                \begin{umlcall}[op=consultaHoras(espec),return=horasMédicos]{Controlador}{Médicos}
                \end{umlcall}
            \end{umlcall}
        \end{umlcall}
    \end{umlcall}
    \begin{umlcall}[op=muestraHoras()]{Funcionario}{Paciente}
    \end{umlcall}
    \begin{umlcall}[op=seleccionaHora()]{Paciente}{Funcionario}
        \begin{umlcall}[op=agendar({hora,médico})]{Funcionario}{Sistema}
            \begin{umlcall}[op=agendar({hora,médico}),return=éxito]{Sistema}{Controlador}
                \begin{umlcall}[op=agendar({hora,médico}),return=éxito]{Controlador}{Médicos}
                \end{umlcall}
            \end{umlcall}
        \end{umlcall}
    \end{umlcall}
    \begin{umlcall}[op=confirmaHora()]{Funcionario}{Paciente}
    \end{umlcall}
    \end{umlseqdiag}
}

\end{ProbPart}


\end{NewAnswer}

\end{spacing}
\end{document}
