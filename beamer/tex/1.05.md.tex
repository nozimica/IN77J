## Narrativa de Casos de Uso

### Narrativa de Casos de Uso

- Los casos de uso no solamente se presentan en la forma de diagramas.
- Una vez que se tiene claro los actores y cómo éstos interactúan con el sistema, se hace
necesario ahondar en esas interacciones:

\begin{rboxx}{}
    \textcolor{blue}{Hay que detallar cada Caso de Uso.}
\end{rboxx}

- Esto se consigue a través de la narrativa de cada uno de ellos.

### Narrativa de Casos de Uso

\subsubtitleB{Características}

- Una narrativa de un caso de uso es una descripción de la historia completa de lo que hacen los actores
a medida que llevan a cabo un caso de uso.
    - Esto termina cuando los actores involucrados logran su objetivo.
- Corresponden a la herramienta más aplicada en la actualidad para especificar los requerimientos funcionales.
- Dado que son una descripción textual, no se requiere de mayores conocimientos técnicos para comprenderlos.

### Narrativa de Casos de Uso

\subsubtitleB{Cómo escribirlos}

- Primero se identifica a los \textcolor{blue}{actores}.
- Luego se identifican las necesidades de esos actores.
    - Esto permite visualizar los casos de uso que ellos requieren.
    - Se detectan tanto las necesidades de los actores como sus eventuales objetivos.
- Estas necesidades dan origen a los casos de uso.
- No es necesario detectar todos los casos de uso al primer análisis. Éstos van visualizándose
a medida que se avanza en el desarrollo del proyecto.
    - Primero se especifica los casos de uso más importantes.
    - Se avanza con los otros de manera incremental.

### Narrativa de Casos de Uso

\subsubtitleB{Cómo escribirlos}

- A medida que se afinan las características de cada caso de uso, se pasa por los siguientes
niveles:

\begin{rboxx}[100mm]{}
\begin{description}
    \item[Identificado o detectado:] sólo se tiene su nombre y los actores asociados.
\end{description}
\end{rboxx}

\begin{rboxx}[100mm]{}
\begin{description}
    \item[Especificado en alto nivel:] se incorpora una sinopsis.
\end{description}
\end{rboxx}

\begin{rboxx}[100mm]{}
\begin{description}
    \item[Especificado de forma extendida:] se incorpora su historia de uso y sus
    eventuales variantes.
\end{description}
\end{rboxx}


### Ejemplo de Narrativa de Casos de Uso

\vspace{-1em}
\begin{tiny}
\begin{usecase}

\addtitle{Caso de uso \#}{Hacer pedido} 

%Scope: the system under design
\addfield{Alcance:}{Atención a público}

%Level: "user-goal" or "subfunction"
% \addfield{Level:}{User-goal}

%Primary Actor: Calls on the system to deliver its services.
\addfield{Actor principal:}{Cliente}
% \addfield{Actores secund:}{}

\addfield{Sinopsis:}{El cliente deberá tener la posibilidad de realizar un pedido, entregando
todos los datos necesarios para poder ser procesado.}

\addfield{Precondiciones:}{Se le presenta al cliente el menú disponible.}
\addfield{Postcondiciones:}{\bld{Garantía}: El pedido queda registrado si el pedido es completado.}

\addscenario{Escenario principal:}{
    \item Usuario \underline{revisa el catálogo e indica sus opciones}.
    \item Se le presenta el monto total de la compra.
    \item Usuario indica forma de pago.
    \item Se procesa la forma de pago.
    \item Usuario recibe comprobante.
}

\addscenario{Escenario alternativo:}{
    \item[5.a] Pago rechazado.
}

\end{usecase}
\end{tiny}

### Elementos en una Narrativa

\subsubtitleB{Número y nombre:}

- Indican el nombre del caso de uso, y también se suele indicar un número, para maś fácil referencia.

\subsubtitleB{Alcance:}

- Corresponde al entorno en el cual se desenvuelve este caso de uso dentro del sistema.
- En sistemas pequeños, este elemento se puede omitir.

### Elementos en una Narrativa

\subsubtitleB{Actores principales y secundarios:}

- Es el listado de los actores principales involucrados en este caso de uso.
    - Los actores principales son aquellos cuyos objetivos son satisfechos directamente
    por el caso de uso.
    - No necesariamente son quienes inician el caso de uso.
- Opcionalmente, se puede segmentar algunos actores como secundarios.
    - Si no se incluyen actores secundarios, simplemente omitir esa línea.


### Elementos en una Narrativa

\subsubtitleB{Sinopsis o resumen:}

- Es simplemente el resumen de lo que hace este caso de uso.
- Es importante que sea breve.
    - Resúmenes muy largos no los leen todas las personas involucradas.
    - El detalle del caso de uso va realmente en el ``escenario principal''.

### Elementos en una Narrativa

\subsubtitleB{Precondiciones:}

- Son las condiciones necesarias para que el caso de uso se lleve a cabo.
    - Si no son satisfechas, el caso de uso no se ejecuta.
- Deben ser reglas conocidas por todos los actores.
- En etapas posteriores del proyecto, estas precondiciones son útiles para
saber qué cosas no hay que revisar dentro del presente módulo.

### Elementos en una Narrativa

\subsubtitleB{Postcondiciones:}

- Describen las principales consecuencias ---para el sistema--- luego de que el caso de uso haya
concluido con éxito.
- Se suele incorporar la palabra \textcolor{blue}{Garantizar}, para hacer énfasis en que
esas condiciones deben estar garantizadas si el caso concluyó correctamente.
    - También se puede indicar garantías mínimas, que son las cubiertas en todos los escenarios.


### Elementos en una Narrativa

\subsubtitleB{Escenario principal:}

- Es el elemento más importante dentro del caso de uso.
- Corresponde a la descripción \textcolor{blue}{paso a paso} de cómo se lleva la interacción
entre el o los actores y el sistema.
- Cada paso debe estar descrito de la manera más concisa posible.
- Debe quedar claro quién está ejecutando cada paso.
- Nunca perder de vista que nos enfocamos en el \textcolor{blue}{qué}, no en el \textcolor{blue}{cómo}.
- Denominado también como ``escenario típico'' o escenario ideal.

### Elementos en una Narrativa

\subsubtitleB{Escenario principal:}

- Alguno de estos pasos puede corresponder a su vez a otro caso de uso.
    - En este caso, estamos frente a lo que ya definimos como \itt{include}.
    - Para indicar que este es el caso, se puede subrayar el paso correspondiente.


### Elementos en una Narrativa

\subsubtitleB{Escenario alternativo:}

- En algunos casos es útil identificar algunos pasos ---dentro del escenario principal--- que
pueden tener un resultado distinto a aquél que se considera como esperable.
- Se mencionan indicando el número del paso correspondiente, adjuntando una letra correlativa.
- Denominado también como ``extensión''.



## Modelado del dominio

### Qué es el modelado del dominio

\subsubtitleB{Definición:}

El modelado del dominio es un modelo conceptual que describe los aspectos más importantes de un
problema:

- Conceptos significativos.
- Relaciones básicas entre conceptos.
- Define un vocabulario mínimo.\newline

Se basa en que la raíz del \texthigh{análisis} orientado a objetos es la descomposición de un problema
en \texthigh{conceptos individuales}.

### Qué es el modelado del dominio

\subsubtitleB{Para qué sirve:}

- \bld{Introducción:} Breve descripción que sirve como introducción al modelo.
- \bld{Lista de conceptos:} Los conceptos que serán las clases de nuestro sistema.
- \bld{Lista de tipos:} Eventualmente presentará los tipos de datos que describan
las propiedades de las clases.
- \bld{Relaciones:} Presenta las relaciones de asociación o generalización entre las clases.


### Qué es el modelado del dominio

\subsubtitleB{Conceptos:}

Se pueden usar símbolos muy parecidos a los de diagramas de clases:

\centering\scalebox{0.6}{%
\begin{tikzpicture}
    \umlclass[type={domain class},x=0,y=0]{Nombre del dominio}{- atributos}{}
\end{tikzpicture}
}

- Muchas veces se omite el estereotipo \itt{<<<domain class>>>}.
    - Sobre todo cuando está claro que tenemos en frente un modelo de dominio.
- Se incluye principalmente los atributos de cada dominio.
    - Los métodos por lo general no se abordan en esta etapa.

### Qué es el modelado del dominio

\subsubtitleA{Conceptos:}{identificación}

- Se comienza la construcción del modelo de dominio haciendo una \texthigh{lista de conceptos
candidatos}.
    - Técnicas comunes:
        - Lista de categorías de conceptos.
        - Identificación de sustantivos.

Repasaremos a continuación cada una de ellas.

### Conceptos

\subsubtitleB{Lista de categorías de conceptos:}

- Para cada categoría dentro de una lista de categorías típicas, se trata de identificar
conceptos relacionados a ellas.
    - Han de ser relevantes para lo que estamos modelando.

\begin{center}
\begin{footnotesize}
\begin{tabular}{lcl}
\toprule
\bld{Categoría} & & \bld{Conceptos} \\
\midrule
Objetos físicos o tangibles & & Avión \\
Descripciones de cosas & & DescripciónDeVuelo \\
Lugares & & Aeropuerto \\
Transacciones & & Reserva \\
Roles & & Piloto \\
\bottomrule
\end{tabular} 
\end{footnotesize}
\end{center}

### Conceptos

\subsubtitleB{Lista de categorías de conceptos:}

He aquí una lista más exhaustiva de las categorías más típicas:

\begin{center}
\begin{footnotesize}
\begin{tabular}{l}
\toprule
\bld{Categoría} \\
\midrule
Contenedores de cosas \\
Cosas contenidas en contenedores \\
Sistemas externos \\
Sustantivos abstractos \\
Organizaciones \\
Eventos \\
Reglas y políticas \\
Catálogos \\
\bottomrule
\end{tabular} 
\end{footnotesize}
\end{center}

### Conceptos

\subsubtitleB{Identificación de sustantivos:}

- Dada una \texthigh{descripción textual} del problema, se identifican en ella
los \texthigh{sustantivos} relevantes.
    - Esta descripción puede ser una narrativa de caso de uso o alguna descripción
    más informal.
- Luego de este paso, hay que filtrar los conceptos realmente significativos.

\subsubtitleB{Ejemplo:}

- \itt{Un \texthigh{cliente} llega a un \texthigh{puesto de venta} para reservar un \texthigh{pasaje} de \texthigh{avión}.}
- \itt{El \texthigh{empleado} hace la \texthigh{reserva} en el \texthigh{sistema de la aerolínea}.}


### Conceptos

\subsubtitleB{Sugerencias:}

- Generar un \texthigh{listado de conceptos} utilizando cualquiera de las dos técnicas recién presentadas.
- Incluir estos conceptos como los \texthigh{primeros elementos} del diagrama de dominio.
- Identificar las \texthigh{asociaciones relevantes} entre estos conceptos.
- Para cada concepto, identificar los \texthigh{atributos necesarios} para satisfacer los requerimientos de
información.
    - Cada atributo en este modelo representa \texthigh{información} que ha de ser \texthigh{necesaria}.

### Conceptos

\subsubtitleB{Importante tener en cuenta:}

- ¿Granularidad fina o compleja?
    - Se prefiere \texthigh{muchos} conceptos de \texthigh{granularidad fina}, que pocos pero muy complejos.
- Es común no identificar todos los conceptos en esta etapa.
    - Una vez descubiertos, se actualiza el modelo de dominio.
- Podremos encontrar conceptos que \texthigh{carezcan de atributos}.
    - Esto indica que no necesariamente guardarán información sobre su estado.

### Conceptos

\subsubtitleB{Errores comunes:}

- Representar como atributo algo que en realidad es un concepto en sí mismo:
    - Si un atributo identificado no es fácilmente representable como un número o un texto,
    hay que pensarlo dos veces: puede ser un concepto.
    - En caso de duda, se puede representar de inmediato como un concepto.
    - O se puede dejar una anotación, para revisarlo en una etapa posterior del análisis,
    cuando se tenga la idea global mucho más clara.

### Conceptos

\subsubtitleA{Errores comunes:}{ejemplo}

Por ejemplo este vuelo:

\tikzinlinec{
    \umlclass[x=0,y=0]{Vuelo}{- código \\ - destino}{}
}

En realidad debiera ser:\newline

\tikzinlinec{
    \umlclass[x=-4]{Vuelo}{- código}{}
    \umlclass[x=4]{Aeropuerto}{- nombre}{}
    \umlassoc[name=finaliza]{Vuelo}{Aeropuerto}
    \draw (finaliza-1) {} node[above=0.2cm] (finaliz) {finaliza en};
    \draw [-triangle 45,above=2mm] (finaliz)+(1.2,0) -- ++(1.3,0);
}

### Conceptos

\subsubtitleB{Errores comunes:}

Analicemos esta situación, aparentemente muy específica:

\columnsbegin

\column{.8\textwidth}

- Se tiene una clase \bld{Producto}, que representa un producto concreto dentro de una tienda.
- Cada instancia de un producto tiene sus propios atributos.
    - Éstos no aparecen en ninguna otra parte.
    - Cuando un producto termina su vida en nuestro sistema (por ejemplo es vendido), la
    instancia que lo representa es eliminada.
        - Esto debe ser así, de lo contrario acumularíamos una cantidad descontrolada de objetos.

\column{.2\textwidth}

\tikzinlinec{
    \umlclass[x=0,y=0]{Producto}{- serie\\ - código \\ - tipo\\ - nombre\\ - precio}{}
}

\columnsend

### Conceptos

\subsubtitleB{Errores comunes:}

- Al no haber más productos de un determinado tipo, sus características habrán sido perdidas para siempre.
    - No se podrá saber cuál era su precio.
    - Tampoco cuál era su descripción.
- Por lo tanto, necesitamos alguna manera de preservar esta información.
    - Independiente de que por ahora no queden productos de aquel tipo.

En este caso, necesitamos una \texthigh{descripción de producto} como un concepto propiamente tal.


### Conceptos

\subsubtitleA{Errores comunes:}{ejemplo}

- Originalmente tendríamos:

\tikzinlinec{
    \umlclass[x=0,y=0]{Producto}{- serie\\ - código \\ - tipo\\ - nombre\\ - precio}{}
    \umlclass[x=7,y=0]{p1 : Producto}{- serie = ``A228''\\ - código = ``34-46d''\\ - tipo = ``Libro'' \\ - nombre = ``El túnel''\\ - precio = 10.000}{}
    \umlclass[x=13,y=0]{p2 : Producto}{- serie = ``A234''\\ - código = ``34-46d''\\ - tipo = ``Libro'' \\ - nombre = ``El túnel''\\ - precio = 10.000}{}
}

- En cambio, podemos modelar esta situación como:

\tikzinlinec[0.5]{
    \umlclass[x=-1,y=0]{Producto}{- serie}{}
    \umlclass[x=4,y=0]{DescProd}{- código \\ - tipo\\ - nombre\\ - precio}{}
%
    \umlassoc[mult1=*,pos1=0.1,mult2=1,pos2=0.9,name=especifica]{Producto}{DescProd}
    \draw (especifica-1) {} node[above=2mm] (espec) {descrito};
    \draw [-triangle 45,above=2mm] (espec)+(1,0) -- ++(1.1,0);
%
    \umlclass[name=p1,x=9,y=1]{p1 : Producto}{- serie = ``A228''}{}
    \umlclass[name=p2,x=9,y=-1.5]{p2 : Producto}{- serie = ``A234''}{}
    \umlclass[name=dp0,x=15,y=0]{ep1 : DescProd}{- código = ``34-46d''\\ - tipo = ``Libro'' \\ - nombre = ``El túnel''\\ - precio = 10.000}{}
    \umlassoc{p1}{dp0}
    \umlassoc{p2}{dp0}
}

### Asociaciones

\subsubtitleB{Conceptos clave:}

- Para identificar las asociaciones entre conceptos hay que tener en cuenta:
    - Que relación y/o dependencia existe entre los conceptos.
    - Qué relación es necesaria para satisfacer los requerimientos de información.
- Siempre identificarán una relación entre conceptos que refleje una conexión relevante o interesante entre ellos.
- La relación puede no ser constante, sino que estará activa por un período de tiempo.
    - Esto no implica que la relación no exista.

### Asociaciones

\subsubtitleB{Lista de categorías de asociaciones:}

- Al igual que con los conceptos, se puede utilizar un listado de categorías de
asociaciones típicas para ayudarnos en el proceso de identificación.

\begin{center}
\begin{footnotesize}
\begin{tabular}{lcl}
\toprule
\bld{Categoría} & & \bld{Conceptos involucrados} \\
\midrule
A es una parte física de B & & Ala --- Avión \\
A es una parte lógica de B & & Tramo --- Ruta \\
A está contenido físicamente en B & & Pasajero --- Avión \\
A está contenido lógicamente en B & & Vuelo --- PlanDeVuelo \\
A es un miembro de B & & Piloto --- Aerolínea \\
\bottomrule
\end{tabular} 
\end{footnotesize}
\end{center}

### Asociaciones

\subsubtitleB{Lista de categorías de asociaciones:}

He aquí una lista más exhaustiva de las categorías más típicas:

\begin{center}
\begin{footnotesize}
\begin{tabular}{l}
\toprule
\bld{Categoría} \\
\midrule
A es una descripción de B \\
A es un ítem de una transacción de B \\
A es conocido/registrado/capturado en B \\
A es una subunidad organizacional de B \\
A usa o maneja a B \\
A se comunica con B \\
A coopera con B \\
A es una propiedad de B \\
\bottomrule
\end{tabular} 
\end{footnotesize}
\end{center}

