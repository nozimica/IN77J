### Relación entre D.C. y D.S.

Entre los diagramas de clases y de secuencia tenemos:

- Ya sabemos que los objetos no son entes estáticos.
    - No solamente guardan datos o propiedades de cada instancia.
    - También \bld{interactúan} entre sí o con instancias de otras clases.

- La interacción se lleva a cabo a través de \alert{mensajes}.
    - Una instancia, al recibir un mensaje, la procesa mediante una operación (implementada en
    un método).
    - Al conjunto de mensajes que una instancia pueda antender, se le denomina \alert{interfaz}.
        - \bld{Ojo} con los distintos usos de la palabra interfaz.

### Ejemplo de diagrama:

\tikzinlinec[0.5]{%
    \umlclass{Cliente}{--nombre : String}{}
    \umlclass[x=8]{Orden}{--fecha : Date\\--estado : int}{+calcImpuesto() : double\\+calcTotal() : double}
    \umlclass[x=5,y=-3]{Pago}{--monto : double}{}
    \umlclass[x=0,y=-8]{Crédito}{--numTarj : String\\--tipo : int\\--expiracion : Date}{+validar() : boolean}
    \umlclass[x=5,y=-8]{Efectivo}{}{}
    \umlclass[x=10,y=-8]{Cheque}{--titular : String\\--banco : String}{+validar() : boolean}
    \umlclass[x=15,y=-8]{DetalleOrden}{--cantidad : int}{+calcSubTotal() : double}
    \umlclass[x=19,y=-8]{Item}{--peso : double\\--descripcion : String}{}
    \umlinherit{Crédito}{Pago}
    \umlinherit{Efectivo}{Pago}
    \umlinherit{Cheque}{Pago}
    \umlaggreg{Orden}{DetalleOrden}
    \umluniassoc{Orden}{Pago}
    \umlassoc{Cliente}{Orden}
    \umluniassoc{Item}{DetalleOrden}
}
