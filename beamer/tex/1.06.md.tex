## Ejemplo: Juego de Dados

### Ejemplo: Juego de Dados

\subsubtitleB{Ejemplo introductorio: Juego de Dados}

Vamos a modelar un juego de dados.\newline

- Tenemos dos dados.
- El usuario cada vez que juegue, lanzará ambos dados.
- Cada vez que lo haga, deseará recibir el valor de cada dado.

### Dados: requerimientos funcionales

\label{dados:clases}
\subsubtitleB{Ejemplo introductorio: Juego de Dados}

- Hay que diseñar un \alert{sistema} que simule el \alert{juego de dos dados}.
    - Este juego consiste en el \alert{lanzamiento} dos \alert{dados}, primero uno y después el otro.
- Está involucrado un solo \alert{jugador}.
- Cada dado, al ser lanzado, da como \alert{resultado} un número en el rango [1, 6].

### Dados: requerimientos funcionales

\subsubtitleB{Ejemplo introductorio: Juego de Dados}

- El juego consiste solamente en el lanzamiento de los dados:
    - Comienza cuando el jugador lanza el \alert{primer dado}.
    - Termina después de obtener el resultado del \alert{segundo dado}.
- Durante su ejecución, el juego debe informar al \alert{usuario} ---a través de una \alert{interfaz gráfica}--- de lo siguiente:
    - Cuándo se inicia el juego.
    - Cuándo se lanza cada dado.
    - El valor obtenido por cada dado.
    - Cuándo el juego termina.

### Dados: casos de uso

\subsubtitleB{Ejemplo introductorio: Juego de Dados}

Para este caso, el sistema simplemente está compuesto por un solo \textcolor{blue}{caso de uso}:

- Jugar Dados.

Y a su vez, hay un solo \textcolor{blue}{actor} involucrado:

- El jugador

\exUseCaseDices

### Dados: narrativa de casos de uso

\subsubtitleB{Ejemplo introductorio: Juego de Dados}

\vspace{-1em}
\begin{tiny}
\begin{usecase}

\addtitle{Caso de uso 1}{Jugar Dados} 

\addfield{Alcance:}{Módulo principal}

\addfield{Actor principal:}{Jugador}

\addfield{Sinopsis:}{El jugador podrá lanzar dos dados, recibiendo el valor de cada uno de ellos.}

\addfield{Precondiciones:}{Se le presenta al jugador la opción de lanzar dados.}
\addfield{Postcondiciones:}{\bld{Garantía}: El jugador recibirá el resultado de los dos dados.}

\addscenario{Escenario principal:}{
    \item Jugador lanza los dados.
    \item Sistema indica que el juego ha comenzado.
    \item Sistema lanza el primer dado.
    \item Sistema muestra el resultado del primer dado.
    \item Sistema lanza el segundo dado.
    \item Sistema muestra el resultado del segundo dado.
    \item Sistema indica que el juego ha finalizado.
}

\end{usecase}
\end{tiny}

### Dados: modelado del dominio

\subsubtitleB{Ejemplo introductorio: Juego de Dados}

Para nuestro ejemplo de los dados:\newline

\centering\scalebox{0.6}{%
\begin{tikzpicture}
    \umlclass[x=0,y=0]{Jugador}{- nombre}{}
    \umlclass[x=9,y=0]{Dado}{- valor}{}
    \umlclass[x=0,y=-4]{Juego}{- fecha}{}
    \umlassoc[mult1=1,pos1=0.1,mult2=2,pos2=0.9,name=tira]{Jugador}{Dado}
    \draw (tira-1) {} node[above=0.2cm] (tr) {tira};
    \draw [-triangle 45,above=2mm] (tr)+(1,0) -- ++(1.1,0);
    \umlassoc[mult1=1,pos1=0.1,mult2=1,pos2=0.9,name=juega]{Jugador}{Juego}
    \draw (juega-1) {} node[right=0.2cm] (jga) {juega};
    \draw [-triangle 45,above=2mm] (jga)+(0,-0.4) -- ++(0,-0.5);
    \umlassoc[geometry=-|,mult1=1,pos1=0.1,mult2=2,pos2=1.9,name=incluye]{Juego}{Dado}
    \draw (incluye-2) {} node[right=0.2cm] (ncly) {incluye};
    \draw [-triangle 45,above=2mm] (ncly)+(1,0) -- ++(1.1,0);
\end{tikzpicture}
}

### Dados: clases candidatas

\subsubtitleB{Ejemplo introductorio: Juego de Dados}

Tomando en cuenta las clases identificadas en la página \ref{dados:clases}, tenemos:

\begin{itemize}
    \item  \tikz\node[inlineBlock] (s1) {Sistema           };
    \item  \tikz\node[inlineBlock] (s2) {Juego de dos dados};
    \item  \tikz\node[inlineBlock] (s3) {Dados             };
    \item  \tikz\node[inlineBlock] (s4) {Primer dado       };
    \item  \tikz\node[inlineBlock] (s5) {Segundo dado      };
    \item  \tikz\node[inlineBlock] (s6) {Resultado         };
    \item  \tikz\node[inlineBlock] (s7) {Lanzamiento       };
    \item  \tikz\node[inlineBlock] (s8) {Jugador           };
    \item  \tikz\node[inlineBlock] (s9) {Usuario           };
    \item  \tikz\node[inlineBlock] (s10){Interfaz gráfica  };
\end{itemize}

\pause

\begin{tikzpicture}[overlay]
  \node[miniconcept, right=50mm of s1] (juegoClass) { Clase Juego };
  \draw[->, thick] ([xshift=1mm]s1.east) -- ([xshift=-2mm]juegoClass.west);
  \draw[->, thick] ([xshift=1mm]s2.east) -- ([xshift=-2mm,yshift=-2mm]juegoClass.west);
%
  \node[miniconcept, right=50mm of s4] (dadoClass) { Clase Dado, y dos instancias };
  \draw[->, thick] ([xshift=1mm]s3.east) -- ([xshift=-2mm,yshift=2mm]dadoClass.west);
  \draw[->, thick] ([xshift=1mm]s4.east) -- ([xshift=-2mm,yshift=0mm]dadoClass.west);
  \draw[->, thick] ([xshift=1mm]s5.east) -- ([xshift=-2mm,yshift=-2mm]dadoClass.west);
%
  \node[miniconcept, right=50mm of s6] (resultadoAttr) { Atributo de Dado };
  \draw[->, thick] ([xshift=1mm]s6.east) -- ([xshift=-2mm]resultadoAttr.west);
%
  \node[miniconcept, right=20mm of s7] (lanzamientoAcc) { Método de Dado };
  \draw[->, thick] ([xshift=1mm]s7.east) -- ([xshift=-2mm]lanzamientoAcc.west);
%
  \node[miniconcept, right=20mm of s9] (jugadorClass) { Instancia de la clase Jugador };
  \draw[->, thick] ([xshift=1mm]s8.east) -- ([xshift=-2mm,yshift=2mm]jugadorClass.west);
  \draw[->, thick] ([xshift=1mm]s9.east) -- ([xshift=-2mm]jugadorClass.west);
%
  \node[miniconcept, right=50mm of s10] (interfazClass) { Instancia de la clase Pantalla };
  \draw[->, thick] ([xshift=1mm]s10.east) -- ([xshift=-2mm]interfazClass.west);
\end{tikzpicture}

